\documentclass{article}
\usepackage[utf8]{inputenc}
\usepackage{amsmath}
\usepackage[T2A]{fontenc}
\usepackage[russian,english]{babel}
\usepackage{amsfonts}
\usepackage{amssymb}
\usepackage{fancyhdr}
\usepackage{graphicx}
\usepackage{tikz}
\usepackage[left=2cm,right=2cm,top=2cm,bottom=2cm]{geometry}
\numberwithin{equation}{section}
\def\letus{%
    \mathord{\setbox0=\hbox{$\exists$}%
             \hbox{\kern 0.125\wd0%
                   \vbox to \ht0{%
                      \hrule width 0.75\wd0%
                      \vfill%
                      \hrule width 0.75\wd0}%
                   \vrule height \ht0%
                   \kern 0.125\wd0}%
           }%
}
\begin{document}
\pagestyle{empty}
\vspace*{9cm}
\begin{center}
\begin{tabular}{|l|l|}
\hline
Лектор & Андреева Татьяна Алексеевна \\ \hline
Дисциплина & Математика \\ \hline
Ученая степень & к.ф.-м.н. \\ \hline
Факультет (специальность) & ИСиТ \\ \hline
Тип & Конспект лекций \\ \hline
Курс (семестр) & 1 курс, 2 семестр \\ \hline
\end{tabular}\\
\vspace*{9cm}
Санкт-Петербург\\
2019
\end{center}
\newpage
\pagestyle{fancy}
\fancyhead[L]{МАТЕМАТИКА}
\fancyhead[R]{Дифференциальные уравнения}
\section{Лекция 11.02.2019}
\textbf{\Large{Глава 1 Дифференциальные уравнения (Д.У.)}}
\\\\
\textbf{\large{\S1 Дифференциальные уравнения первого порядка}}
\\\\
\textbf{1. Общие понятия}
\\\\
Д.У. первого пордяка называется уравнение вида
\begin{equation}\label{e1}
F(x,y,y')=0,
\end{equation}
где x -- аргумент,\\
y -- неизвестная функция\\
Чаще всего рассматривается уравнение относительно производной
\begin{equation}\label{e2}
y'=f(x,y)
\end{equation}
иногда Д.У. первого порядка записывают в виде
\begin{equation}\label{e3}
P(x,y)dx+Q(x,y)dy=0
\end{equation}
в этом случае за неизвестную функцию можно принять как x так и y.
$$P+Q\cdot y'=0, y'=-\dfrac{P}{Q}$$
$$P\cdot x'+Q=0, y'=-\dfrac{Q}{P}$$
Функция $y=f(x)$ называется решением уравнения \ref{e1}, если она
$$F(x,\varphi(x),\varphi'(x))=0$$
Решения, заданные неявно, то есть в виде $\varphi(x)=0$ называется интегралом дифференциального уравнения.\\
\textit{Пример}\\
Доказать, что функция $\varphi(x)=e^{\sqrt{1-x^2}}$ есть решение уравнения $xydx+\sqrt{1-x^2}dy=0$\\
Решение: положив $y=e^{\sqrt{1-x^2}}$, найдем $dy=\dfrac{-xe^{\sqrt{1-x^2}}}{\sqrt{1-x^2}}dx$ и подставив значения $y$ и $dy$ в данные уравнения получим тождество $xe^{1-x^2}dx+\sqrt{1-x^2}\cdot\dfrac{-xe^{\sqrt{1-x^2}}}{\sqrt{1-x^2}}\equiv0$
\\\\
\textbf{2. Дифференциальные уравнения семейства кривых}
\\\\
Однопарметрическим семейстовом кривых называется совокупность линий, определяемая уравнением:
\begin{equation}\label{e4}
\varphi(x,y,c)=0
\end{equation}
Фиксируя значение параметра $c$, получают конкретную линию данного семейства. Так например, уравнение $y=cx^2$, определяет собой семейство паробол с вершиной в начале координат, симметричных относительно оси $Oy$. Придавая параметру $c$ значения 1, -1, 2 получаем параболы $y=x^2$, $y=-x^2$, $y=2x^2$\\\\
\begin{tikzpicture}
\draw[->] (-2,0)--(2,0) node [right] {$x$};
\draw[->] (0,-2)--(0,2) node [above] {$y$};
\node at (1.5,1) {$c=1$};
\node at (1.5,-1) {$c=-1$};
\draw (-1,-2) .. controls (-0.5,0.65) and (0.5,0.65) .. (1,-2);
\draw (-1,2) .. controls (-0.5,-0.65) and (0.5,-0.65) .. (1,2);
\end{tikzpicture}\\\\
Дифференцируя семейство уравнений по $x$: получим частные производные
\begin{equation}\label{e5}
\dfrac{\partial\varphi(x,y,c)}{\partial x}+\dfrac{\partial\varphi(x,y,c)}{\partial y}y'=0
\end{equation}
и исключая из двух уравнений \ref{e4} и \ref{e5} параметр $c$, мы придем к дифференциальному уравнению, вида $F(x,y,y')=0$, которому удовлетворяет любая линия данного семейства\\
\textit{Пример}\\
из семейства окружностей $x^2+y^2=c^2$ выделить, ту которая проходит через точку $A(3,4)$\\
Решение: Чтобы выделить нужную, надо найти соответствующее ей значение параметра $c=c_0$. Так как искомая окружность проходит через точку $M$, то координаты этой точки удовлетворяют $x^2+y^2=c_0^2$. Подставляя значения $x=3$, $y=4$, получим $c_0^2=25\Rightarrow x^2+y^2=25$\\
\textit{Пример}\\
Составить дифференциальное уравнение семейства окружностей $x^2+y^2=c^2$ и установить дифференциальные свойства этого семейства.\\
Решение: продифференцируем уравнение $x^2+y^2=c^2$ по $x$: $2x+2yy'=0$ или $y'=-\dfrac{x}{y}$, не содержащее параметра $c$ оно будет искомым. Из $y'=-\dfrac{x}{y}$ следует, что угловой коэффициент касательной к окружности, проходящей через точку $M(x,y)$ равен $-\dfrac{1}{k}$, где $k$ -- угловой коэффициент касательной к прямой $OM$
$$k=\tg\beta\dfrac{y}{x}, y'=\tg\alpha=-\dfrac{x}{y}$$
$$\tg\beta=\dfrac{-1}{\tg\alpha}, \alpha=\dfrac{\pi}{2}+\beta$$
\begin{tikzpicture}
\draw[->] (-2,0)--(2,0) node [right] {$x$};
\draw[->] (0,-2)--(0,2) node [above] {$y$};
\node at (1,1) {$M$};
\node at (1.5,0) [above right]{$\alpha$};
\node at (0,0) [below right] {$\beta$};
\draw (0,0) circle (1);
\draw (0,0) circle (0.5);
\draw[dashed] (0,1.4)--(1.4,0);
\draw (0,0)--(0.7,0.7);
\draw (1.5,0) arc (0:135:0.1);
\draw (0.1,0) arc (0:45:0.1);
\end{tikzpicture}\\
\\\\
\textbf{3. Графическое истолкование Д.У.}
\\\\
Пусть $y=\varphi(x)$ решение Д.У. $F(x,y,y')=0$. График функции $y=\varphi(x)$ называется игнтегральной кривой уравнения $F(x,y,y')=0$. Само Д.У. утсанавливает зависимость между координтами точки $(x,y)$ и укловым коэффициентом касательной $y'$ к интегральной кривой в той же точке. Рассомтрим Д.У. \ref{e2}, разрешенное относительно $y'$, где $f(x,y)$ задана в области $(D)$ на плоскости $xOy$. В каждой точке $(x,y)$ области $(D)$ уравнения \ref{e2} указывает направление касательной к интегральной кривой, проходящей через эту точку. Тем самым, геометрически, уравнение \ref{e2} равносильно заданию в $(D)$ поля напрвлений, а интегрирование этого уравнения равносильно проведению таких линий, которые в каждой своей точке касаются направления поля, заданного в этой точке (направляющей поля, заданной в этой точке)\\\\
\begin{tikzpicture}
\draw[->] (-1,0)--(3,0) node [right] {$x$};
\draw[->] (0,-1)--(0,3) node [above] {$y$};
\node at (0,0) [below left] {0};
\node at(2.5,2.5) {$(D)$};
\draw (1.5,1.5) circle (1);
\draw[->] (1,1)--(1.5,1.5);
\draw[->] (1,1.5)--(1.5,2);
\draw[->] (1.5,1)--(2,1.5);
\end{tikzpicture}
\begin{tikzpicture}
\draw[->] (0,0) .. controls (2,1) and (2,-1) .. (4,0);
\draw[->] (1,0.3)--(1.5,0.5) node [right] {$(x_0,y_0)$};
\end{tikzpicture}\\\\
Такое геометрическое истолкование Д.У. позволяет проинтегрировать его графически, то есть представить его приблеженную картину ряда интегральных кривых для этого нужно покрыть область $(D)$ густой сеткой точек и в каждой выбранной точке $(x_0,y_0)$ начертить небольшую стрелку наклоненную под углом $\alpha$ к оси $Ox$, где $\tg=f(x_0,y_0)$. Затем проведем ряд интегральных кривых, по направлениям указанным стрелочками. Обычно, при этом пользуются методом изоклин. Изоклиной называется линия, вдоль которой направление поля определяемого диф уравнением \ref{e2}, одно и тоже. Уравнение изоклины получается из уравнения \ref{e2}, если положить $y'=const$, то есть
\begin{equation}\label{e6}
f(x,y)=c
\end{equation}
Придавая уравнению \ref{e6} параметр $c$ ряд значений строят несколько изоклин и на каждой из них наносят ряд стрелочек или штрихов наклоненных к оси $Ox$ под углом $\alpha$, для которого $\tg\alpha=c$ по направлению этих стрелочек и проводят интегральные кривые.
\\\\
\textbf{4. Задача Коши. Теорема существования и единственности решения уравнения $y'=f(x,y)$}
\\\\
Задача Коши для Д.У. первого порядка состоит в том чтобы найти решение, которое при заданном значении аргумента $x=x_0$ принимает заданное значение $y=y_0$, то есть удовлетворяет начальному условию $y|_{x=x_0}=y_0$. Среди всех интегральных кривых данного Д.У. выделить ту которыя проходит через заданную точку $(x,y)$.\\\\
Решение называют частным решением Д.У.\\
Функция $y=\varphi(x,c)$, где $c$ -- проивзольная постоянная, называется общим решением Д.У.в некоторой области $(D)$ $xOy$, если соответствующем выборе значения $c$ эта функция обращается в $\forall$ частное решение, график которго лежит в области $(D)$. Уравнение $\Phi(x,y,c)=0$ называется общим интегралом данного Д.У. В области $(D)$, если при соответствующем выборе значения $c$ оно определит любую интегральную прямую в области $(D)$.
\section{Лекция 18.02.2019}
\textbf{Теорема существования и единственности решения.}\\
Пусть дано Д.У. $y'=f(x,y)$ и начальное условие $y(x_0)=y_0$, то есть точка $(x_0,y_0)$. Если в некотором прямоугольнике $R(a\leqslant x\leqslant b, c\leqslant y\leqslant d)$, содержащем внутри себя точку $(x_0,y_0)$ функция $f(x,y)$ непрерывна по $x$ и $y$ и имеет ограниченную частную производную по $y$, то существует его единственное решение $y=y(x)$ этого уравнения, определенное в интеграле $a<x<b$ и принимающее при $x=x_0$ значение $y(x_0)=y_0$.\\\\
\begin{tikzpicture}
\draw[->] (-1,0)--(3,0);
\draw[->] (0,-1)--(0,3);
\node at (0,0) [below left] {0};
\draw[dashed] (0.5,2.5)--(0.5,0) node [below] {a};
\draw[dashed] (2.5,2.5)--(2.5,0);
\draw[dashed] (2.5,2.5)--(0,2.5);
\draw[dashed] (2.5,0.5)--(0,0.5);
\draw (0.5,1) .. controls (1.5,1) and (2,2.5) .. (2.5,2.5);
\filldraw (1.9,2) circle (1pt) node [left] {y=y(x)};
\end{tikzpicture}\\\\
Другими словами при любых начальных условиях внутри прямоугольника $(R)$ существует единственное им соответстующее решение, если $f(x,y)$ удовлетворяет наложенным условиям.
Геометрически это означает, что через каждую внутреннюю точку $(x_0,y_0)$ проходит только одна интегральная прямая.\\
Теорема существования и единственности решения применяется при решении прикладных задач, так как в соответствии с ней, найдя решение, удовлетворяющее заданным ограниченным условиям, можно быть уверенным, что других решений нет. Получая единственный закон явления, который определяется дифференциальным уравнением и начальным условиям.\\\\
\textbf{Общий вид}\\\\
\textit{1. Интегрирование Д.У., не содержащих искомую функцию}\\\\
$y'=f(x)$\\
Предположим, что $f(x)$ определена и непрерывна в $(a,b)$, так как $y'=\dfrac{dy}{dx}$, то данное уравнение можно записать в виде:\\
$\dfrac{dy}{dx}=f(x)$ -- разделяя переменные:\\
$dy=f(x)dx$ -- правая часть -- дифференциал некоторой функции $F(x)$ поэтому интегрируя $\int dy=\int f(x)=dx\Rightarrow y=F(x)+c$\\
\colorbox{red!50}{***}\\
\textit{Пример}\\
проинтегрировать Д.У. $y'=3x-1$\\
$y'=\dfrac{dy}{dx}\Rightarrow\dfrac{dy}{dx}=3x-1\Rightarrow dy=(3x-1)dx$\\
$\int dy=\int(3x-1)dx$\\
$\Rightarrow y=\int(3x-1)dx=3\int xdx-\int dx=\dfrac{3x^2}{2}-x+c$\\\\
\textit{2. Интегрирование Д.У. с различными переменными} $f(x)dx+\varphi(y)dy=0$, где $f(x)$ и $\varphi(y)$ -- коэффициенты при $dx$ и $dy$ -- явлюятся непрерывными функциями соответственно только по $x$ или только по $y$. В этом случае Д.У. можно интегрировать почленно. Общий интеграл: $\int f(x)dx+\int\varphi(y)dy=c$ или $\int\limits_{x_0}^xf(x)dx+\int\limits_{y_0}^y\varphi(y)dy=c$. Если $f(x_0)$ и $\varphi(y_0)$ не равны одновременно нулю, то решение при Н.У. $(x_0,y_0)$ можно найти обычным способом.\\\\
\textit{3. Интегрирование Д.У. с разделяющимися переменными}\\\\
\begin{equation}\label{e7}
f(x)\cdot\varphi(y)dx+\Psi(x)\Phi(y)dy=0\footnote{Д.У. первого порядка с разделяющимися переменными}
\end{equation}
Пусть $f(x), \varphi(y), \Psi(x), \varPhi(y)$ непрерывны\\
Разделим Д.У. \eqref{e7} на произведение $\varphi(y)\Psi(x)$:\\
$\dfrac{f(x)dx}{\Psi(x)}+\dfrac{\varPhi(y)dy}{\varphi(y)}=0$\\
Обозначим $F(x)=\dfrac{f(x)}{\Psi(x)}, F(y)=\dfrac{\varPhi(y)}{\varphi(y)}$\\
Получим $F(x)dx\cdot F(y)dy=0$\\
Интегрируя почленно последнее уравнение получем его общий интеграл, а следовательно и общий интеграл исходного уравнения.\\
$\int F(x)dx+\int F(y)dy=c$ или $\int\limits_{x_0}^xF(x)dx+\int\limits_{y_0}^yF(y)dy=c$\\
Обозначим через $H(x)$ и $G(x)$ первообразные интегралов, входящих в уравнение \colorbox{red!50}{(2)} и получим: $G(y)=-H(x)+c$. Разрешив последнее равенство относительно $y$ найдем общее решение Д.У.: $y=K(x)+C_*$\\
$K(x)$ -- символическая запись получаемой новой функции $x$, а $C_*$ новая постоянная интегрирования.\\
\textit{Пример}\\
Решить Д.У. $1\cdot xdy=1\cdot ydx\bigg|\cdot\dfrac{1}{xy}$\footnote{уравнение с разделяющимися переменными}\\
$\ln y=\ln x+\ln C$\\
$\ln y=\ln Cx\Rightarrow y=Cx$\\\\
\textit{4. Интегрирование однордных Д.У.}\\\\
Функция $f(x,y)$ называется однородной, если $f(tx,ty=t^nf(x,y)$ $f(x,y)=\sqrt{x^2+y^2}$\\
$f(tx,ty)=\sqrt{t^2x^2+t^2y^2}=\sqrt{t^2(x^2+y^2)}=t\sqrt{x^2+y^2}$\\
Д.У. вида: $M(x,y)dx+N(x,y)dy=0$ называется однородным, если $M(x,y)$ и $N(x,y)$ -- однородные функции своих аргументов одинаковой степени, непрервных вместе со своими частными производными первого порядка. Однородное Д.У. преобразовывается в Д.У. с разделяющимися переменными. Вводим новую функцию $\dfrac{y}{x}=v$ или $y=xv$. Дифференциируя, находим: $dy=vdx+xdv$\\
\textit{Пример}\\
$xy+y^2=(2x^2+xy)y'$ $(xy+y^2)dx=(2x^2+xy)dy$\\
$M(x,y)=xy+y^2\Rightarrow M(tx,ty)=txty+(ty)^2=t^2xy+t^2y^2=t^2(xy+y^2)$\\
$xy+y^2=(2x^2+xy)y'\bigg|y=vx$\\
$vx^2+v^2x^2=x^2(2+v)\dfrac{d(vx)}{dx}\Rightarrow v+v^2=(2+v)(\dfrac{dv}{dx}x+v)\Rightarrow v+v^2=2x\dfrac{dv}{dx}+vx\dfrac{dv}{dx}+2v+v^2\Rightarrow x(2+v)\dfrac{dv}{dx}=-v\bigg|\dfrac{dx}{x\cdot v}$\\
$\dfrac{2+v}{v}dv=-\dfrac{dx}{x}\Rightarrow 2\ln v+v+\ln x=c\Rightarrow\ln(v^2x)=c-v$\\
Потенцируя\footnote{$e^{ln(v^2x}=v^2x$}, получаем: $v^2x=e^{c-v}$ или $v^2x=c_1e^{-v}$, где $c_1=e^v$. То есть $v=\dfrac{y}{x}$, то $\dfrac{y^2}{x^2}=c_1e^{-\frac{y}{x}}\Rightarrow$ общий интеграл: $y^2=c_1xe^{-\frac{y}{x}}$\\\\
\textit{5. Интегрирование линейных Д.У.}\\\\
Л.Д.У. 1-ого порядка -- уравнение, разрешенное относительно производной $y'$ вида:
\begin{equation}\label{e9}
\dfrac{dy}{dx}+Py=Q
\end{equation}
где $P$ и $Q$ -- функции $x$ или постоянные величины. Иными словами Л.Д.У. есть уравнение 1 степени относительно одной из переменных (в данном случае $y$) и ее производной. Неизвестная функция $y$ входит линейно то есть в 1 степени при этом исходное уравнение \eqref{e9} в котором правая часть равна 0, называется однородным (не путать с однородными Д.У.), а у которого правая часть не равна 0 называется неоднородным.
\section{Лекция 25.02.2019}
Пускай функция $P(x)$ и $G(x)$ определены и непрерывны в $(a,b)$. Тогда через каждую точку полосы $a<x<b$ $-\infty<y<+\infty$ проходит одна и только одна интегральная кривая уравнения \colorbox{red!50}{\eqref{e1}}, которая определена во всем интервале $(a,b)$. Следовательно, в задаче Коши для Л.Д.У. $x_0$ -- любое число на $(a,b)$, а $y_0$ -- любое число.\\\\
\textit{I. Метод Бернулли (подстановка Бернулли) для решения Л.Д.У. 1-ого порядка}\\\\
\begin{equation}\label{e10}
y'+P(x)\cdot y=Q(x)
\end{equation}
\begin{equation}\label{e11}
y=u(x)\cdot v(x)=u\cdot v
\end{equation}
$$y'=\dfrac{dy}{dx}$$
\begin{equation}\label{e12}
dy=udv+vdu
\end{equation}
\begin{equation}\label{e13}
\dfrac{dy}{dx}=u\dfrac{dv}{dx}+v\dfrac{dy}{dx}
\end{equation}
$$\dfrac{dy}{dx}+P(x)\cdot y=Q(x)\Rightarrow$$
\begin{equation}\label{e14}
\Rightarrow dy+P(x)\cdot ydx=Q(x)dx\mbox{ (применяя подстановку Бернулли)}
\end{equation}
\begin{equation}\label{e15}
udv+vdu+P(x)uvdx=Q(x)dx
\end{equation}
\begin{equation}\label{e16}
u(dv+P(x)vdx)+vdu=Q(x)dx
\end{equation}
\begin{equation*}
\begin{cases}
dv+P(x)vdx=0\\
vdu=Q(x)dx
\end{cases}
\end{equation*}
$$\dfrac{dv}{v}=-P(x)dx$$
Сгруппируем, например, второй и третий члены в \colorbox{red!50}{***} равенства:
\begin{equation}\label{e17}
udv+v(du+P(x)udx)=Q(x)dx
\end{equation}
Поставим условие: найти такую функцию $u$, чтобы выражение в скобках обратилось в 0, такое требование обосновано, так как пока функция $u(x)$ -- производная и на нее не наложено никаких ограничений.
\begin{equation}\label{e19}
du+P(x)udx=0
\end{equation}
После раздления переменных Д.У. для определения $u$: $\dfrac{du}{u}=-P(x)dx\Rightarrow\ln u=-\int P(x)dx$. Потенцируя, получим:
\begin{equation}\label{e33}
u=e^{-\int P(x)dx}
\end{equation}
Постоянную интегрирования не пишем, так как достаточно будет какого-нибудь отличного от 0 решения Д.У. \eqref{e19}. Подставим найденную функцию в уравнение \eqref{e17}. Так как выражение в скобках равно 0, получим $udv=Q(x)dx$ или $e^{-\int P(x)dx}dv=Qdx\Rightarrow dv=Q(x)e^{\int P(x)dx}dx$. Интегрируя последнее уравнение:
\begin{equation}\label{e20}
v=\int Q(x)e^{\int P(x)dx}dx+c\mbox{ (нашли вторую неизвестную функцию)}
\end{equation}
Подставляя найденные $u$ и $v$ в исходную подстановку Бернулли: $y=u(x)v(x)$, получим искомое общее решение:
\begin{equation}\label{e21}
y=e^{-\int P(x)dx}(\int Q(x)e^{-\int P(x)dx}+c)
\end{equation}
\textit{Пример}\\
проинтегрировать Л.Д.У. методом Бернулли: $xy'-3y=x^2$\\
$y'+P(x)y=Q(x)$ $y=uv\Rightarrow y'=u'v+v'u$\\
$xu'v+xuv'-uv=x^2$\\
Объединим 2-ой и 3-ий члены: $(xv'-3v)u+xu'v=x^2$\\
Приравнивая выражение в скобках, получим: $xv'-3v=0\Rightarrow\dfrac{dv}{v}-3\dfrac{dx}{x}=0\Rightarrow\dfrac{dv}{v}=3\dfrac{dx}{x}\Rightarrow\ln v=3\ln x\Rightarrow v=x^3$\\
При таком выборе функции $v$ преобразованное Д.У. запишется: $xu'x^3=x^2$ или $u'x^2=1\Rightarrow x^2du=dx$ и $du=\dfrac{dx}{x^2}\Rightarrow u=-\dfrac{1}{x}+c\Rightarrow y=uv=x^3(c-\dfrac{1}{x})=cx^3-x^2$\\\\
\textit{II способ решения Л.Д.У. 1-ого порядка: Метод Лагранжа или метод вариации постоянной}\\\\
\begin{equation}\label{e22}
y'+P(x)y=Q(x)
\end{equation}
Общее решение для однородного Л.Д.У. \eqref{e22} находим с помощью общего решения соответствующего однородного уравнения варьируя в нем произвольную постоянную то есть считая ее некоторой непрерывной дифференцируемой функцией от $x$. Найдем общее решение однородного Л.Д.У. соответствующего \eqref{e22}.
\begin{equation}\label{e23}
y'+P(x)y=0
\end{equation}
После разделения переменных: $\dfrac{dy}{y}=-P(x)dx\Rightarrow\ln y=-\int P(x)dx+c_1\Rightarrow y=e^{-\int P(x)dx+c_1}=e^{-\int P(x)dx}\cdot e^{c_1}$. Полагая для кратности $e^{c_1}=c$, получаем общее решение однородного Л.Д.У.
\begin{equation}\label{e24}
y=ce^{-\int P(x)dx}
\end{equation}
или в форме Коши:
\begin{equation}\label{e25}
y=y_0e_0e^{-\int\limits_{x_0}^xP(x)dx}
\end{equation}
Положим: $y=c(x)ce^{-\int P(x)dx}$, где $c(x)$ -- некоторая непрерывно дифференцируемая функция $x$. Для определения $c(x)$ подставим \eqref{e24} в \eqref{e22}. Тогда учитывая, что $\dfrac{dy}{dx}=c'(x)e^{-\int P(x)dx}+c(x)e^{-\int P(x)dx}\cdot(-\int P(x)dx)=c'(x)e^{-\int P(x)dx}-P(x)c(x)e^{-\int P(x)dx}$. Далее: $c'(x)e^{-\int P(x)dx}-P(x)c(x)e^{-\int P(x)dx}+P(x)c(x)e^{-\int P(x)dx}=Q(x)$ или после сокращения: $c'(x)e^{-\int P(x)dx}=Q(x)$\\
Выполняя очевидные преобразования получим: $dc(x)e^{-\int P(x)dx}=Q(x)dx$ или $dc(x)=Q(x)e^{\int P(x)dx}dx$. Интегрируя, получим: $c(x)=\int Q(x)e^{\int P(x)dx}dx+c$. В результате подстановки полученного $c(x)$ в \colorbox{red!50}{***} находим общее решение данного неоднородного Л.Д.У.: $y=e^{-\int P(x)dx}(\int Q(x)e^{\int P(x)dx})dx+c$, которое совпадает с О.Р. (общим решением), полученным подстановкой Бернулли.\\
\textit{Пример}\\
Проинтегрировать методом вариации постоянных (методом Лагранжа) неоднородные Л.Д.У.: $y'+y\ctg x=\tg x$
\section{Лекция 04.03.2019}
\textit{Пример}\\проинтегрировтаь методом Лагранжа: $x(x^2-1)y'-(2x^2-1)y=-ax^3\bigg|\dfrac{1}{x(x^2-1)}\bigg|$\\
$y'-\dfrac{2x^2-1}{x(x^2-1)}y=-\dfrac{ax^3}{x(x^2-1)}=0$\\
1) Решаем уравнение: $y'-\dfrac{2x^2-1}{x(x^2-1)}y=0$\\
$\dfrac{dy}{y}=\dfrac{2x^2-1}{x(x^2-1)}dx$\\
$$\int\dfrac{dy}{y}=\int\dfrac{2x^2-1}{x(x^2-1)}dx=\int\bigg(\dfrac{1}{x}+\dfrac{2x}{2(x^2-1)}\bigg)dx\Rightarrow\ln y=\ln x+\dfrac{1}{2}\ln(x^2-1)+\ln c=\ln cx\sqrt{x^2-1}\Rightarrow$$
потенцируя полученно равенство. Общее решение однородного Д.У. $y=cx\sqrt{x^2-1}$. Для решения неоднородного Д.У. применяем метод вариации постоянных полагая $c=c(x)$, тогда $y=c(x)x\sqrt{x^2-1}=c=c(x)(x^4-x^2)^\frac{1}{2}(4x^3-2x)=c'(x)x\sqrt{x^2-1}+c(x)\dfrac{2x^2-1}{\sqrt{x^2-1}}$. Полученные выражения для $y$ и $y'$ подставляем в исходное неоднородное Д.У.\\
$c'(x)x\sqrt{x^2-1}+cx\dfrac{2x^2-1}{\sqrt{x^2-1}}-\dfrac{2x^2-1}{x(x^2-1)}c(x)x\sqrt{x^2-1}=-\dfrac{ax^2}{x(x^2-1)}$\\
Полагаем: $c(x)\dfrac{2x^2-1}{\sqrt{x^2-1}}-\dfrac{2x^2-1}{x(x^2-1)}c(x)x\sqrt{x^2-1}=0$, получаем:\\
$c'(x)x\sqrt{x^2-1}=-\dfrac{ax^3}{x(x^2-1)}$\\
Тогда: $c'(x)=\dfrac{ax^2}{x(x^2-1)\sqrt{x^2-1}}=-\dfrac{ax^3}{\sqrt{x(x^2-1)}}$ или\\
$c(x)=-\int\dfrac{ax^3}{\sqrt{x^2-1}}dx=\dbinom{\mbox{подстановка}}{x^2-1=z\Rightarrow2xdx=dz}=c(x)=-\dfrac{a}{2}\int\dfrac{dz}{\sqrt{z^3}}=-\dfrac{a}{2}\int z^{-\frac{3}{2}}dz=-\dfrac{a}{2}\dfrac{z^{-\frac{1}{2}}}{-\frac{1}{2}}+c_1=\dfrac{a}{\sqrt{z}}+c_1=\dfrac{a}{\sqrt{x^2-1}}+c_1\Rightarrow y=\bigg(\dfrac{a}{\sqrt{x^2-1}}+c_1\bigg)(x^4-x^2)^{\frac{1}{2}}$ -- общее решение неоднородного уравнения\\
Интегрирование Д.У. в полных дифференциалах. Если в Д.У.
\begin{equation}\label{e26}
M(x,y)dx+N(x,y)dy=0
\end{equation}
Если в таком Д.У. есть левая часть -- есть полный дифференциал от некой функции $u(x,y)$, то такое уравнение нащывается Д.У. полного дифференциала или $M(x,y)dx+N(x,y)dy\equiv du(x,y)$ в этом случае уравнение \eqref{e26} можно представить в виде $du(x,y)=0$ и его общий интеграл: $u(x,y)=c$. Пусть $M(x,y)$ и $N(x,y)$ определены и непрерывны в некоторой односвязанной области $(D)$ и имеют в ней непрерывные частные производные по $x$ и $y$. Необходимым и достаточным условием для того, чтобы уравнение \eqref{e26} было Д.У, в полных дифференциалах является выполнение тождества:
\begin{equation}\label{e27}
\dfrac{dM}{dy}=\dfrac{dN}{dx}
\end{equation}
Если условие \eqref{e27} выполнено, то общий интеграл можно записать в виде
\begin{equation}\label{e28}
\int\limits_{x_0}^xM(x,y)dx+\int\limits_{y_0}^yN(x_0,y)dy=c\mbox{, или}
\end{equation}
\begin{equation}\label{e29}
\int\limits_{x_0}^xM(x,y_0)dx+\int\limits_{y}^yN(x_0,y)dy=c
\end{equation}
где точка $(x_0,y_0)$ принадлежит области $(D)$. Здесь интегрирование производится по одной переменной, а другая является параметром.\\
Решение задачи Коши \colorbox{red!50}{***} с начальными параметрами $x_0, y_0$ в области $(D)$, при условии, что в точке $(x_0,y_0)$ функции $M(x,y)$ и $N(x,y)$ не обращаются одновременно в 0, получается из общего интеграла \eqref{e28} и \eqref{e29} при $c=0$ $\int\limits_{x_0}^xM(x,y)dx+\int\limits_{y_0}^yN(x_0,y)dy=0$ или $\int\limits_{x_0}^xM(x,y_0)dx+\int\limits_{y_0}^yN(x,y)dy=0$. Если уравнение \eqref{e26} не является уравнением полным дифференциалом, то существует такая дифференцируемая функция $\mu=\mu(x,y)$, что вот такое уравнение $\mu(Mdx+Ndy)=0$ которое уже представляет собой Д.У. с полным дифференциаломю.  Функция $\mu$ называется интегрируемым множителем уравнения \eqref{e26}, если это так, то $\dfrac{\partial(\mu M)}{\partial y}=\dfrac{\partial(\mu N)}{\partial x}$, тогда условие \colorbox{red!50}{***}. Если интегрирующий множитель $\mu$ определен, то интегрирование Д.У. сводится к умножению этого уравнения на $\mu$ и определению общего интеграла полученного уравнения в полных дифференциалах. Из последнего тождества следует что $\mu$ удовлетворяет Д.У. в частных производных, а именно:
\begin{equation}\label{e30}
N\dfrac{\partial M}{\partial x}-M\dfrac{\partial N}{\partial y}=\mu\bigg(\dfrac{\partial M}{\partial y}-\dfrac{\partial N}{\partial x}\bigg)
\end{equation}
Определение интегрирующего множителя невсегда возможно и если заранее известно $\mu=\mu(\nu)$, где $\nu$ -- заданная функция $x$ и $y$, то уравнение \eqref{e30} приводится к Л.Д.У. с неизвестной (искомой) функцией $\mu$ от аргумента $\nu$:
\begin{equation}\label{e31}
\dfrac{\partial\mu}{\partial\nu}=\varphi(\nu)\cdot\mu
\end{equation}
\begin{equation}\label{e32}
\dfrac{\dfrac{\partial M}{\partial y}-\dfrac{\partial N}{\partial x}}{N\dfrac{\partial\nu}{\partial x}-M\dfrac{\partial\nu}{\partial x}}\equiv\varphi(\nu)
\end{equation}
Решая Д.У. \eqref{e31} находим $\mu=e^{\int\varphi(\nu) d\nu}$\\
\textit{Пример}\\
проинтегрировать Д.У. $(3x^2+6xy^2)dx+(6x^2y+4y^3)dy=0$\\
Решение: в данном Д.У. частная производная $\dfrac{\partial M(x,y)}{\partial y}=(3x^2+6xy^2)'_y=12xy$ и $\dfrac{\partial N(x,y)}{\partial x}=(6x^2y+4y^3)'_x=12xy$ то есть $\dfrac{\partial M}{\partial y}\equiv\dfrac{\partial N}{\partial x}-12xy\Rightarrow$ условие тождественности \eqref{e27} выполнено и исходное уравнение является Д.У. в полных дифференциалах. Группируя члены в Л.У. данного уравнения: $3x^2dx+6xy(ydx+xdy)+4y^3dy=0$\\
Так как $3x^2dx=d(x^3)$\\
$6xy(ydx+xdy)=6yd(xy)=d(3(xy)^2)$\\
$4y^3dy=d(y)$\\
Тогда данное Д.У. можно записать в виде: $d(x^3)+d(3(xy)^2)+d(y^4)=0$ или $\underbrace{d(\overbrace{x^3+3x^2y^2+y^4}^{u(x,y)})}_{du=0}=0$ или $x^2+3x^2y^2+y^4=c$
\\\\
\textbf{\Large{Глава 2 Д.У. высших порядков}}
\\\\
\textbf{\large{\S1 Основные понятия и определения}}
\\\\
О.Д.У. $n$-ого порядка -- уравнение связывающее неизвестную функцию $y$, независимую переменную $x$ и производную функции $y$ по $x$, до $n$-ого порядка включительно.
\begin{equation}\label{e34}
F(x,y,y',y'', ... , y^{(n)})=0
\end{equation}
Ограничимся рассмотрением уравнения $n$-ого порядка разрешенным относительно старшей производной.
\begin{equation}\label{e35}
y^{(n)}=f(x,y,y',y'',...,y^{(n-1)})
\end{equation}
где $f$ -- определенная, однозначная, непрерывная функция в области \colorbox{red!50}{изменения} своих аргументов.\\
Решением \eqref{e35} в интервале $(a,b)$ называется функция $y=\varphi(x)$, которая обращает это уравнение в тождество при всех значениях $x$, причем имеет в заданном интервале непрерывные производные до $n$-ого порядка включительно и начальные значения $x, \varphi'(x), \varphi''(x), ..., \varphi^{(n)}(x)$ принадлежат области задания функции $f$ для всех значений $x$ на данном интервале. Решение Д.У. содержит $n$ произвольных постоянных $c$ и представляет не одну, а целое семейство интегральных кривых зависящих от $n$ параметров: $c_1,c_2,c_3,...,c_n$. Для определения постоянных необходимо задать $n$ условий. Общий интеграл уравнения \eqref{e35} выглядит: $\Phi(x,y)=0$. Д.У. \eqref{e35} называется линейным, если его правая часть является линейной функцией относительно аргументов $y,y',y'',...,y^{(n-1)}$. Другими словами
\begin{equation}\label{e36}
y^{(n)}+p_1(x)y^{(n-1)}+p_2(x)y^{(n-2)}+...+p_{n-1}(x)y'+p_n(x)y=f(x)
\end{equation}
где $p_i(x)(i=1,2,...,n)$ -- непрерывные рациональные коэффициенты, зависящие в общем случае от $x$, $f(x)$ -- непрерывная функция в $(a,b)$ функция $x$.
\\\\
\textbf{\large{\S2 Задача Коши. Теорема $\exists!$ решения}}
\\\\
Среди всех решений \eqref{e35} найти такое решение $y=y(x)$ для которого $y(x)$ вместе c $(n-1)$ последовательными производными принимает заданные значения $y_0,y'_0,...,y_0^{(n-1)}$, при заданном значении $x_0$ независимой переменной $x$, то есть:
\begin{equation}\label{e37}
y(x_0)=y_0;y'(x_0)=y'_0;y''(x_0)=y''_0;...,y^{(n-1)}(x_0)=y_0^{(n-1)}
\end{equation}
где $x_0,y_0,y'_0,y''_0,...,y_0^{(n-1)}$ -- заданные числа, причем при $x=x_0$ решение $y=y(x)$ удовлетворяет условиям:
\begin{equation}\label{e38}
\left.
\begin{array}{l}
y=y_0\\
y'=y'_0\\
y''=y''_0\\
...\\
y^{(n-1)}=y_0^{(n-1)}\\
\mbox{Числа }y_0,y'_0,y''_0,...,y_0^{(n-1)}\mbox{-- начальные значения решения }y=y(x)
\end{array}
\right\}
\end{equation}
Число $x_0$ -- значение независимой переменной\\
Числа $x_0,y_0,y'_0,...,y_0^{(n-1)}$ в совокупности называются начальными данным решения. Условие \eqref{e37} -- начальные условия данного решения.\\
Рассмотрим Д.У. второго порядка: $y''=f(x,y,y')$\\
Задача Коши для этого уравнения сотоит в в определении решения $y=y(x)$, удовлетворяющего следующим начальным условиям: при $x=x_0$ $y=y_0$ $y'=y'_0$\\
Геометрически решение этого Д.У. сводится к нахождению интегральной прямой, проходящей через заданную точку $P(x_0,y_0)$ и имеющую в ней заданную касательную, которая образует с $OX$ угол $\alpha$ $\alpha_0=$acrtg $y_0$
\section{Лекция 11.03.2019}
---------------------------------------------------------------\colorbox{red!50}{здесь должен быть рисунок?}---------------------------------------------------------------
\\\\
Геометрически решение этого Д.У. сводится к нахождению интегральной кривой через заданную точку и имеющий в ней заданную касательную, которая образует \colorbox{red!50}{***}\\
Th $\exists!$ решения задачи Коши определяет условия при которых решение Д.У. существует и является единственным при заданных начальных условиях. Задача Коши с начальными условиями имеет решение, если правая часть уравнения $y^{(n)}=f(x,y_0,y'_0,y''_0,...,y_0^{(n-1)})$ непрерывна в окрестности начальных значений. Для единственности решения также необходимо, чтобы правая часть Д.У. имела ограниченные частные производные по $y,y',y'',...,y^{(n-1)}$\\
Строгая формулировка:\\
$\letus$ дано Д.У. $y(n)=f(x,y,y',y'',...,y^{(n-1)})$.\\
Заданы начальные условия: $y=y_0,y'=y'_0,y''=y''_0,...,y^{(n-1)}=y_0^{(n-1)}$ при $x=x_0$\\
Предположим, что $f(x,y,y',...,y^{(n-1)}$ определена в некоторой замкнутой ограниченной области $(D)$:\\
$|x-x_0|\leqslant a, |y-y_0|\leqslant b, |y'-y'_0|\leqslant b$\\
$|y^n-y^n_0|\leqslant b,...,|y^{(n-1)}-y_0^{(n-1)}|\leqslant b$\\
С начальными значениям: $(x_0,y_0,y'_0,...,y_0^{(n-1)})$ внутри этой области, где $a$ и $b$ -- заданные положительные числа и удовлетворяют в этой области условиям:\\
1) Функция непрерывна по всем аргументам $u\Rightarrow$ ограничена, то есть $|f(x,y,y',y'',...,y^{(n-1)}\leqslant M$, где $M$ -- положительная константа.\\
2) Функция имеет ограниченные частные производные по всем аргументам $\bigg|\dfrac{\partial f(x,y,y',y'',...,y^{(n-1)})}{\partial y^{(i)}}\bigg|\leqslant k$ $(i=0,1,2,...,(n-1); y^{(0)}=y)$, где $k$ -- произвольное положительное число\\
Тогда данное Д.У. имеет единственное решение $y=y(x)$ удовлетворяет Н.У.\footnote{начальные условия}
\\\\
\textbf{\large{\S3 Общее и частное решение}}
\\\\
О.Р.Д.У.\footnote{Общее решение Д.У.}
\begin{equation}\label{e39}
y^{(n)}=f(x,y,y',y'',...,y^{(n-1)})
\end{equation}
в области $(D)$ изменения переменных $x,y,y',y'',...,y^{(n-1)}$ для которой $\exists!$ решение задачи Коши называется функция
\begin{equation}\label{e40}
y=\varphi(x_1,c_1,...,c_n)
\end{equation}
определенная в некоторой области изменения переменных $x,c_1,c_2,...,c_n$ и имеющая частные производные по $x$ до $n$-ого порядка включительно, если:\\
а) система Д.У.:
\begin{equation}\label{e41}
\begin{array}{l}
y=\varphi(x,c_1,c_2,...,c_n)\\
y'=\varphi'(x,c_1,c_2,...,c_n)\\
...\\
y^{(n-1)}=\varphi^{(n-1)}(x,c_1,c_2,...,c_n)
\end{array}
\end{equation}
разрешима в области $(D)$ относительно произвольной постоянной
\begin{equation}\label{e42}
\left.
\begin{array}{l}
c_1=\varphi_1(x,y,y',y'',...,y^{(n-1)})\\
c_2=\varphi_2(x,y,y',y'',...,y^{(n-1)})\\
...\\
c_n=\varphi_n(x,y,y',y'',...,y^{(n-1)})
\end{array}
\right\}
\end{equation}
б) функция $y=\varphi(x,c_1,c_2,...,c_n)$ является решением Д.У. $n$-ого порядка \eqref{e39} при всех значениях произвольных постоянных, получаемых из соотношений \eqref{e42}, когда начальные значения: $(x,y,y',y'',...,y^{(n-1)})$ принадлежат области $(D)$.\\
Для нахождения решения Д.У. \eqref{e39} с начальными данными, если известно общее решение \eqref{e40}, следует подставить в систему \eqref{e41} начальные значения, тогда получим систему алгебраических уравнений.\\
\begin{equation}\label{e43}
\left.
\begin{array}{l}
y_0=\varphi(x,c_1,c_2,...,c_n)\\
y_0'=\varphi'(x,c_1,c_2,...,c_n)\\
...\\
y_0^{(n-1)}=\varphi^{(n-1)}(x,c_1,c_2,...,c_n)
\end{array}
\right\}
\end{equation}
Решая эту систему относительно $c_1=c_1^{(0)},c_2=c_2^{(0)},...,c_n=c_n^{(0)}$, подставляем найденные значения найдем общее решение: $y=\varphi(x,c_1^{(0)},c_2^{(0)},...,c_n^{(0)})$\\
Общим решением в форме Коши называется общее решение:
\begin{equation}\label{e44}
y=\varphi(x,x_0,y_0,y_0',y_0'',...,y_0^{(n-1)})
\end{equation}
в котором роль произвольных постоянных играют Н.З.\\
$y_0$ решения $y=y(x)$ и начальные значения $y'_0,y''_0,...,y_0^{(n-1)}$ производных этого решения при фиксированном $x_0$\\
Общим интегралом Д.У. $n$-ого порядка \eqref{e39} называется общее решение этого уравнения, заданного в неявном виде: $\varPhi(x,y,c_1,c_2,...,c_n)=0$\\
Частным решением Д.У. $n$-ого порядка \eqref{e39} называется решение в каждой точке которого сохраняется единственность решения задачи Коши.
\\\\
\textbf{\large{\S4 Фундаментальная система решений}}
\\\\
Общий вид Н.Л.Д.У.\footnote{неоднородное линейное Д.У.} $n$-ого порядка:
\begin{equation}\label{e45}
y^{(n)}+p_1(x)y^{(n-1)}+p_2(x)y^{(n-2)}+...+p_{n-1}(x)y'+p_n(x)y=f(x)
\end{equation}
При $f(x)=0$ уравнение \eqref{e45} принимает вид:
\begin{equation}\label{e46}
y^{(n)}+p_1(x)y^{(n-1)}+p_2(x)y^{(n-2)}+...+p_{n-1}(x)y'+p_n(x)y=0
\end{equation}
$\letus$ коэффициенты $p_1(x),p_2(x),...,p_{n-1}(x),p_n(x)$ и свободный член $f(x)$ являются определенными и непрерывными функциями $x$ в интервале $(a,b)$\\
Тогда Д.У. \eqref{e45} имеет единственное решение $y=y(x)$, определены во всем интервале $(a,b)$ и удовлетворяющее $y=y_0,y'=y'_0,y''=y''_0,...,y^{(n-1)}=y^{(n-1)}_0$ при $x=x_0$\\
Начальные данные можно задавать произвольно, а значение $x_0$ должно быть из интервала $(a,b)$. Всякое решение Л.Д.У. \eqref{e39} является частным решением этого уравнения.\\
\textbf{А} О.Л.Д.У. $n$-ого порядка \eqref{e40} всегда имеет нулевое (тривиальное) решение и нулевое решение также удовлетворяет Н.У.\\
Для нахождения О.Р. О.Л.Д.У. \eqref{e40} необходимо знать $n$ линейно независимых частных решений $y_1,y_2,...,y_{n-1},y_n$ для которых \colorbox{red!50}{***} тождество:
\begin{equation}\label{e47}
\lambda_1y_1+\lambda_2y_2+\lambda_3y_3+...+\lambda_{n-1}y_{n-1}+\lambda_ny_n=0
\end{equation}
где $\lambda_1,\lambda_2,...,\lambda_n$ -- постоянные числа, выполнимо только в следующем очевидном случае $\lambda_1=\lambda_2=...=\lambda_n=0$. Такая система решений О.Л.Д.У. $n$-ого порядка называется фундаментальной системой решений. Для того чтобы система решений $y_1,y_2,y_3,...,y_{n-1},y_n$ была фундаментальной линейнонезависимой необходимо и достаточно, чтобы определитель $W(x)=\begin{vmatrix}
y_1 & y_2 & ... & y_n\\
y_1' & y_2' & ... & y_n'\\
...\\
y_1^{(n-1)} & y_2^{(n-1)} & ... & y_n^{(n-1)}
\end{vmatrix}$\footnote{называемый определитель Вронского (Вронскиан)} был отличен от нуля, то есть $W(x)\neq0(x\in(a,b))$. При условии определенности непрерывности коэффициенты $p_i(x)(i=1,2)$ О.Л.Д.У. $n$-ого порядка \eqref{e46} имеет фундаментальную систему решений и даже бесчисленные множества их.\\
Ф.С.Р.: $y_1,y_2,y_3,...,y_{n-1},y_n$ О.Л.Д.У. \eqref{e46} $n$-ого порядка называется нормированной в точке $x=x_0$, если эти решения удовлетворяют следующим начальным условиям:
\begin{equation}\label{e48}
\left.
\begin{matrix}
y_1=1 & y_1'=0 & y_1''=0 & ... & y_1^{(n-1)}=0\mbox{ при }x=x_0\\
y_2=0 & y_2'=1 & y_2''=0 & ... & y_2^{(n-1)}=0\mbox{ при }x=x_0\\
...\\
y_n=0 & y_n'=0 & y_n''=0 & ... & y_n^{(n-1)}=1\mbox{ при }x=x_0
\end{matrix}
\right\}
\end{equation}
Если известно Ф.С.Р. $y_1,y_2,...y_n$, то общее решение в области $\begin{matrix}
a<x<b\\
-\infty<y<+\infty\\
-\infty<y'<+\infty\\
-\infty<y''<+\infty\\
...\\
-\infty<y^{(n-1)}<+\infty
\end{matrix}$ имеет вид:
\begin{equation}\label{e49}
y=c_1y_1+c_2y_2+c_3y_3+...+c_{n-1}y_{n-1}+c_ny_n    
\end{equation}
где $c_1,c_2,...,c_n$ -- произвольные постоянные\\
\textbf{Б} Н.Л.Д.У. $n$-ого порядка\\
Для нахождения общего решения Н.Л.Д.У. $n$-ого порядка, достаточно найти одно его частное решение и сложить с общим решением соответствующего О.Л.Д.У., то есть если $y_1$ -- частное решение Д.У. \eqref{e45}, а общее решение соответствующего Л.Д.У. имеет вид:
\begin{equation}\label{e50}
z=c_1z_1+c_2z_2+...+c_nz_n
\end{equation}
, то общее решение Н.Л.Д.У. $n$-ого порядка \eqref{e45} такое:
\begin{equation}\label{e51}
y=y_1+c_1z_1+c_2z_2+...+c_nz_n
\end{equation}
Пусть правая часть Н.Л.Д.У. \eqref{e45} $f(x)$ состоит из нескольких слагаемых, тогда если для соответствующих Н.Л.Д.У. правая часть которых равна каждому из слагаемых, смжно найти частное решение, то сумма этих частных решений является частным решением всего Н.Л.Д.У \eqref{e45}. Это свойство обеспечивает нахождение Н.Л.Д.У. $n$-ого порядка.\\
\textbf{В} Н.Л.Д.У. 2-ого порядка
\section{Лекция 25.03.2019}
\textbf{\large{\S1 Линейное Д.У. второго порядка}}\\\\
Общий вид:
\begin{equation}\label{e52}
y''+p(x)y'+q(x)y=0\mbox{ (однородное)}
\end{equation}
где $p(x)$ и $q(x)$ -- непрерывные функции.\\
Общее решение:
\begin{equation}\label{e53}
y=c_1y_1+c_2y_2
\end{equation}
где $y_1$ и $y_2$ -- частные, линейно независимые решения ЛДУ \eqref{e52}.\\
Л.Д.У. второго порядка неоднородное:
\begin{equation}\label{e54}
y''+p(x)y'+q(x)y=f(x)
\end{equation}
где $f(x)$ -- непрерывная функция\\
Общее решение неоднородного Д.У.:\\
$$y=c_1y_1+c_2y_2+\tilde y$$
где $y_1$ и $y_2$ -- линейно независимые решения соответствующего однородного уравнения, $\tilde y$ -- частное решение\\
Д.У. второго порядка, однородное с потоянными коэффициентами
\begin{equation}\label{e55}
y''+a_1y'+a_2y=0
\end{equation}
где $a_1$ и $a_2$ -- постоянные коэфициенты. Будем искать общее решение этого Л.О.Д.У. в виде $y=e^{rx}$\\
Имеем $y'=re^{rx},y''=r^2e^{rx}$\\
После подстановки в исходное уравнение \eqref{e55} получим: $e^{rx}(r^2+a_1r+a_2)=0$ или $r^2+a_1r+a_2=0$\\
Для составления характеристического уравнения Л.Д.У. необходимо каждую производную искомой функции заменить через $r$ в степени равной порядку производной.\\
$(y=y^0,y',y'')$ $(y=r^0,r,r^2)$\\
то есть нахождения частных решений Л.О.Д.У. второго порядка сводиться к решению квадратных уравнений\\
\textit{Случай 1}\\
Корни характеристического уравнения действительные и различные ($r_1\neq r_2$)\\
В этом случае имеем два линейно независимых частных решения: $y=c_1e^{r_1x}$ и $y_2=c_2e^{r_2x}$\\
Общее решение Л.Д.У. $y=c_1y_1+c_2y_2$, то есть $y=c_1e^{r_1x}+c_2e^{r_2x}$\\
\textit{Случай 2}\\
Корни характеристического уравнения комплексные, попарно сопряженные ($r_1=\alpha+\beta i, r_2=\alpha-\beta i$). Принимая во внимание формулу Эйлера: $e^{(\alpha+\beta i)x}=e^{\alpha x}e^{\beta ix}=e^{\alpha x}(\cos\beta x+i\sin\beta x)$\\
Найдем, что общее решение Л.Д.У. \eqref{e55} имеет вид:
\begin{equation}\label{e56}
y=c_1e^{(\alpha+\beta i)x}+c_2e^{(\alpha-\beta i)x}=c_1e^{(\alpha x)}\cos\beta x+c_2e^{\alpha x}\sin\beta x=e^{\alpha x}(c_1\cos\beta x+c_2\sin\beta x)
\end{equation}
Общее решение \eqref{e56} можно представить в тригонометрической форме
\begin{equation}\label{e57}
y=Me^{ax}\sin(bx+\varphi)
\end{equation}
где $M$ и $\varphi$ -- постоянные величины\\
\textit{Случай 3}\\
Корни характеристического уравнения действительные и равные ($r_1=r_2$)\\
Обшее решение Л.Д.У. имеет вид:
\begin{equation}\label{e58}
y=c_1e^{r_1x}+c_2xe^{r_1x}=e^{r_1x}(c_1+c_2x)
\end{equation}
Для решения О.Л.Д.У. с постоянными коэффициентами второго порядка, необходимо:\\
1) составить характеристическое уравнение\\
2) найти его корни\\
3) состовить общее решение, соответствующее типу полученных корней\\
\textit{Пример 1}\\
Проинтегрировать Д.У. $y''-y'-2y=0$\\
Характеристическое уравнение $r^2-r-2=0$\\
Корни $r_1=2$, $r_2=-1$\\
Фундаментальная система решений:
\begin{equation*}
\begin{cases}
y_1=e^{2x}\\
y_2=e^{-x}
\end{cases}
\end{equation*}
Общее решение имеет вид $y=c_1e^{2x}+c_2e^{-x}$\\
\textit{Пример 2}\\
$y''-4y'+13y=0$\\
$r^2-4r+13=0$\\
$r_1=2+3i$, $r_2=2-3i$\\
Здесь $\alpha=2$, $\beta=3$\\
Фундаментальная система решений
\begin{equation*}
\begin{cases}
y_1=e^{2x}\cos3x\\
y_2=e^{2x}\sin3x
\end{cases}
\end{equation*}
Общее решение имеет вид: $y=e^{2x}(c_1\cos3x+c_2\sin3x)$\\
Л.Д.У. второго порядка неоднородное с постоянными коэффициентами
\begin{equation}\label{e59}
y''+a_1y'+a_2y=f(x)
\end{equation}
где $f(x)$ -- заданная непрерывная функция $x$ или (в частном случае) постоянное число. Общее решение Неод.Д.У. $Y$ является суммой общего решения соответствующего Л.Д.У. однородного и одного частного решения Л.Н.Д.У. $y_1$, то есть $Y=y+y_1$. Вид частного решения $y_1$ зависит от вида кривой части Л.Д.У.\\
\textit{Случай 1}\\
Предположим, что Л.Д.У. -- многочлен вида:
\begin{equation}\label{e60}
f(x)=a_0x^m+a_1x^{m-1}+a_2x^{m-2}+...+a_{m-1}x+a_m
\end{equation}
Тогда $y_1=A_0x^m+A_2x^{m-1}+...+A_{m-1}x+A_m$\\
Если характеристическое уравнение имеет корень равный нуля кратности $k(k=1,2)$, то
\begin{equation}\label{e61}
y_1=x^k(A_0x^m+A_1x^{m-1}+...+A_{m-1}x+A_m
\end{equation}
\textit{Случай 2}\\
Предположим, что Л.Д.У. является показательной функцией вида: $f(x)=ae^{px}$, где $a$ -- постоянная величина или многочлен от $x$. Тогда
\begin{equation}\label{e62}
y_1=Ae^{px}
\end{equation}
где $A$ -- постоянная или многочлен от $x$. В случае, если характеристическое уравнение имеет корень $p$ кратности $k(k=1,2)$, то $y_1=Ax^ke^{px}$\\
\textit{Случай 3}\\
Предположим, что Л.Д.У. имеет вид: $f(x)=a_1\sin bx+a_2\cos bx$, тогда $y_1=A_1\sin bx+A_2\cos bx$\\
Если характеристическое уравнение имеет корни: $a\pm bi$, кратности $k$, то $y_1=x^k(A_1\sin bx+A_2\cos bx)$\\
Если правая часть $f(x)=e^{ax}(c_1\sin bx+c_2\cos bx)$, то $y_1=e^{ax}(D_1\sin bx+D_2\cos bx)$\\
Здесь $c_1$ и $c_2$ соответственно $D_1$ и $D_2$ могут быть многочленами от $x$ или постоянными величинами. В случае, если характеристическое уравнение имеет комплексные корни кратности $a+bi$ и $a-bi$, то $y_1=x^ke^{ax}(D_1\sin bx+D_2\cos bx)$\\
\textit{Случай 4}\\
Если правая часть $f(x)$ Д.У. является суммой функций рассмотренных видов, то частное решение $y_1$ равнно сумме соответствующих функций. Коэффициенты $A_0,A_1,...,A_m,D_1,D_2$ -- подлежат определению. Они должны определяться так, чтобы $y_1$ действительно было частным решением данного Л.Д.У. После подстановки $y_1$ в данное уравнение оно должно обращаться в тождество, из которого, сравнение коэффициентов при подобных членах в обеих частях, определяются искомые коэффициенты и многочлены.
\section{Лекция 01.04.2019}
\textit{Пример}\\
Проинтегрировать методом вариации переменных Л.Н.Д.У. третьего порядка с постоянными коэффициентами $y'''-y'=e^{2x}$. Соответстввующее Л.О.Д.У.: $y'''-y'=0$\\
1. Составим характеристическое уравнение: $r^3-r=r(r^2-1)$\\
Корни $r_1=0,r_2=\pm1$ -- действительные различные числа $\Rightarrow$ общее решение:
\begin{equation}\label{e63}
y(x)=c_1e^0+c_2e^x+c_3e^{-x}=c_1\cdot1+c_2e^x+c_3e^{-x}
\end{equation}
Общее решение Л.Н.Д.У.: Полагаем, что произвольной постоянной: $c_1,c_2,c_3$ -- искомые функции $c_1(x),c_2(x),c_3(x)$. Дифференцируя \eqref{e63} получим: $y'=c'_1(x)\cdot1+c_1(x)\cdot0+c'_2(x)e^x+c_2(x)e^x+c'_3(x)e^{-x}-c_3(x)e^{-x}=c'_1(x)\cdot0+c_2(x)e^x+c_3(x)e^{-x}+1\cdot c'_1(x)e^x+c'_2(x)e^x+c_3(x)e^{-x}$\\
Налагаем на неизвестные функции дополнительные условия $c'_1(x)\cdot1+c'_2(x)e^x+c'_3(x)e^{-x}=0\Rightarrow y'=c_1(x)0+c_2(x)e^x-c_3(x)e^{-x}$\\
Вторая производная:
\begin{equation}\label{e64}
\begin{matrix}
y''=c'_1(x)\cdot0+c_1(x)\cdot0+c'_2(x)e^x+c_2(x)e^x-c'_3e^{-x}+c_3(x)e^{-x}=\\
=c_1(x)\cdot0+c_2(x)e^x+c_3(x)e^{-x}+c'_1(x)\cdot0+c'_2(x)e^{x}-c'_3(x)e^{-x}
\end{matrix}
\end{equation}
В полученном выражении \eqref{e64} налагаем условие: $c_1'(x)\cdot0+c_2'(x)e^x-c_3'(x)e^{-x}=0$\\
Тогда $y''=c_1(x)\cdot0+c_2(x)e^x+c_3(x)e^{-x}$\\
Находим третью производную: $y'''=c_1'(x)\cdot0+c_1(x)\cdot0+c_2'(x)e^x+c_2(x)e^x+c_3'(x)e^{-x}-c_3(x)e^{-x}=c_1(x)\cdot0+c_2(x)e^x-c_3(x)e^{-x}+c_1'(x)\cdot0+c_2'(x)e^x+c_3'e^{-x}$\\
Подставляя выражения $y'''$ и $y'$ в данное Д.У. получим: $c_1(x)\cdot0+c_2(x)e^x-c_3e^{-x}+c_1'(x)\cdot0+c_2'(x)e^x+c_3'(x)e^x-c_1(x)\cdot0-c_2(x)e^x+c_3(x)e^{-x}=e^{2x}$, откуда
\begin{equation}\label{e65}
c_1'(x)\cdot0+c_2'(x)e^x+c_3'(x)e^{-x}=e^{2x}
\end{equation}
Для определения неизвестных $c_1'(x),c_2'(x),c_3'(x)$, получаем неоднородную систему алгебраических уравнений \eqref{e63}--\eqref{e65}
\begin{equation}\label{e66}
\left.
\begin{matrix}
c_1'(x)\cdot1+c_2'(x)\cdot e^x+c_3(x)e^{-x}=0\\
c_1'(x)\cdot0+c_2'(x)\cdot e^x-c_3(x)e^{-x}=0\\
c_1'(x)\cdot0+c_2'(x)\cdot e^x+c_3(x)e^{-x}=e^{2x}
\end{matrix}
\right\}
\end{equation}
Определитель системы \eqref{e66}:
$D=\left|
\begin{matrix}
1 & e^x & e^{-x}\\
0 & e^x & -e^{-x}\\
0 & e^x & e^{2x}
\end{matrix}
\right|=1\cdot(e^x\cdot e^x+e^x\cdot e^{-x})=e^0+e^0=2\neq0!$
Решение системы находим по методу Крамера\\
$c_1'(x)=\dfrac{D_1}{D}=\dfrac{1}{2}\left|
\begin{matrix}
0 & e^x & e^{-x}\\
0 & e^x & -e^{-x}\\
e^{2x} & e^x & e^{-x}
\end{matrix}
\right|=\dfrac{e^{2x}}{2}(-e^0-e^0)=-e^{2x}$\\
$c_2'(x)=\dfrac{D_2}{D}=\dfrac{1}{2}\left|
\begin{matrix}
1 & 0 & e^{-x}\\
0 & 0 & -e^{-x}\\
0 & e^{2x} & e^{-x}
\end{matrix}
\right|=\dfrac{1}{2}(0+e^x)=\dfrac{1}{2}e^x$\\
$c_3'(x)=\dfrac{D_3}{D}=\dfrac{1}{2}\left|
\begin{matrix}
1 & e^x & 0\\
0 & e^x & 0\\
0 & e^x & e^{2x}
\end{matrix}
\right|=\dfrac{1}{2}(e^{3x}-0)=\dfrac{1}{2}e^{3x}$\\
$c_1(x)=-\int e^{2x}dx=-\dfrac{1}{2}e^{2x}+c_1^*$\\
$c_2(x)=\dfrac{1}{2}\int e^xdx=-\dfrac{e^x}{2}+c_2^*$\\
$c_3(x)=\dfrac{1}{2}\int e^{2x}dx=-\dfrac{1}{6}e^{3x}+c_3^*$\\
Искомое решение Л.Н.Д.У.: $y(x)=c_1(x)\cdot1+c_2(x)e^x+c_3(x)e^{-x}=c_1^*+c_2^*e^x+c_3^*e^{-x}+(-\dfrac{1}{2}e^{2x}+\dfrac{1}{2}e^{2x}+\dfrac{1}{6}e^{2x})=(c_1^*+c_2^*e^x+c_3^*e^{-x})+\dfrac{1}{6}e^{2x}$, где $c_1^*+c_2^*e^x+c_3e^{-x}$ -- общее решение соответствующего Л.О.Д.У., а $\dfrac{1}{6}e^{2x}$ -- частное решение Л.Н.Д.У.
\\\\
\textbf{\Large{Глава 3 Элементы операционного исчисления (ОИ)}}
\fancyhead[R]{Операционное исчисление}
\\\\
Возникновение элементов операционного исчисления, как самостоятельного раздела математики относится к концу XIX века, хотя истоки ОИ имелись в работах Лейбница, Бернулли, Лагранжа, Эйлера, Коши, Фурье. Сущность ОИ состоит в том, что оператор дифференциального исчисления рассматривается как алгебраическая величина, благодаря чему с ним можно производить действия как с числами. Метод ОИ позволяет посредством простых \colorbox{red!50}{***} решать сложные задачи в математике, физике: в теории специальных функций, при вычислении интегралов и суммировании углов, проблемах теории чисел и т.д. Наибольшое значения оно имеет в автоматике и телемеханике, теории \colorbox{red!50}{***} систем, теории регулирования.
\\\\
\textbf{\large{\S1 Оригинал и изображение}}
\\\\
\textbf{1. Определение оригинала и изображения}
\\\\
Будем рассматривать комплексную функцию действительного аргумента $f(t)=u(t)+iv(t)$\\
\textit{Определение 1}\\
Функцию $f(t)$ удовлетворяющую следующим условиям:\\
1)$f(t)$ и $f'(t)$ или всюду непрерывна или имеют на любом различном промежутке лишь конечное число точек разврыва 1 рода.\\
2)$f(t)=0$ для всех точек $t<0$\\
3)$f(t)$ возрастает не быстрее показательной функции, то есть $\exists$ такие числа $M>0$ и $S_0\geqslant0$, для которых $|f(t)\leqslant Me^{S_0t}|$, $S_0$ -- показатель функции роста функции $f(t)$, тогда $f(t)$ называют начальной функцией или оригиналом.\\
\textit{Пример}\\
Можно ли функцию $f(t)=\begin{cases}
0,t<0\\
e^{(5-4i)t},t>0
\end{cases}$ назвать оригиналом?\\
1. Она непрерывна повсюду, кроме $t=0$ (как и ее производная), при $t<0$ $f(t)=0$, $f(t)=e^{5t}\Rightarrow$ для этой функции $M=1,S_0=5$, поэтому данная функция -- оригинал.\\
Функция Хевисайда: так называется единая функция $\eta(t)=\begin{cases}
0,t<0\\
1,t\geqslant0
\end{cases}$\\\\
\begin{tikzpicture}
\draw[->] (-1,0)--(4,0) node [right] {t};
\draw[->] (0,-1)--(0,2) node [above] {$\eta(t)$};
\node at (0,0) [below left] {0};
\draw (4,1)--(0,1) node [left] {1};
\end{tikzpicture}\\\\
Хевисайд(1850--1925) -- английсикий инженер-физик\\
$\letus$ некоторая функция $\varphi(t)$ удовлетворяет условиям 1 и 3 определения 1, но не удовлетворяет условию 2, то есть $\varphi(t)\neq0$ для значений $t<0$. Умножив эту функцию на $\eta(t)$, мы \colorbox{red!50}{***} $\varphi(t)$ для значений $t<0$ и не изменяем ее для значений $t\geqslant0$. Таким образом $\eta(t)\cdot\varphi(t)=\begin{cases}
0,t<0\\
\varphi(t),t\geqslant0
\end{cases}$\\
Пользуясь единичной функцией можно найти оригиналы $\sin t,\cos t,\ch t,\sh t,e^t$\\
$\eta(t)\cdot\sin t=\begin{cases}
0,t<0\\
\sin t,t\geqslant0
\end{cases}$ будет оригиналом, порядок роста которого $M=1,S_0=0$, так как $|\sin t|\leqslant1$, $t$ -- вещественная переменная.\\
Сущность операционного метода заключается в замене оригинала $f(t)$, такой функцией $F(p)$, при которой операция дифференцирования и интегрирования над оригиналом, заменяется алгебраическими операциями над функцией $F(p)$. Здесь $p=s+i\tau$ -- комплексная переменная величина. Такую функцию $F(p)$ называют изображением функции $f(t)$ и обозначают одним из следующих выражений: $f(t)$\colorbox{red!50}{***}$F(p)$; $f(t)\rightarrow F(p)$; $L\{f(p)\}=F(p)$. Для перехода от оригинала к изображению и наоборот надо знать свойства оригинала и изображения. Зная эти свойства можно составить таблицу-каталог \colorbox{red!50}{***} (таблица изображений и оригиналов функций)\\
\textit{Определение 2}\\
Изображение оригинала $f(t)$ называется функция $F(p)$ связанная с оригиналом равенством 
\begin{equation}\label{e67}
F(p)=\int\limits_0^\infty f(t)\cdot e^{-pt}dt
\end{equation}
где $p=s+i\tau$, путем интегрирования является вещественная положительная полуось. Интеграл в формуле \eqref{e67} называют интегралом Лапласа, для функции $f(t)$, переход от оригинала $f(t)$ к изображению $F(p)$ -- преобразование Лапласа. Теория преобразований Лапласа -- операционным исчислением.
\\\\
\textbf{\large{\S2 Условия существования изображения}}
\\\\
Пусть $f(t)$ -- оригинал в какой-то области \colorbox{red!50}{***} $P$, существует изображение этого оригинала, иначе говоря в какой области $t$ интеграл Лапласа сходится. Ответ на этот вопрос дает теорема существования.\\
\textit{Теорема}\\
Если оригинал $f(t)$ имеет порядок роста $S_0$, то изображение этого оригинала $F(p)$ существует для всех $p$, для которых $Re$ $p=S>S_0$, и является при этом условии аналитической функцией. В теореме утверждается, что интеграл Лапласа сходится и является аналитической функцией в области расположенной в комплексной плоскости $p$ справа от переменной $S=S_0$\\\\
\begin{tikzpicture}
\draw[->] (-1,0)--(4,0) node [right] {S};
\draw[->] (0,-1)--(0,2);
\node at (0,0) [below left] {0};
\node at (1,0) [below right] {$S_0$};
\draw (1,0)--(3,2);
\draw (2,0)--(4,2);
\draw (3,0)--(4,1);
\draw (1,1)--(2,2);
\draw (1,-1)--(1,2);
\end{tikzpicture}\\\\
Сходимость интеграла Лапласа как следует из теоремы определяется величиной $Re$ $p=S$ и не зависит от $Im$ $p=\tau$ -- мнимой части параметра $p$.
\section{Лекция 08.04.2019}
Докажем существование функции $F(p)$: Пусть $Re$ $p=S>S_0$. Оценим модуль интеграла Лапласа при этом условии: $|\int\limits_0^\infty f(t)\cdot e^{-pt}dt|\leqslant\int\limits_0^\infty |e^{-pt}|\cdot|f(t)|dt$\\
По свойству показательной функции: $|e^{-pt}|=e^{-pt}$, и $|\int\limits_0^\infty f(t)\cdot e^{-pt}dt|\leqslant\int\limits_0^\infty e^{-pt}\cdot Me^{S_0t}dt$, то есть модуль интеграла Лапласа можно оценить следующим выражением
$$\Big|\int\limits_0^\infty f(t)\cdot e^{-pt}dt\Big|\leqslant M\int\limits_0^\infty e^{-(S-S_0)t}dt=-M\dfrac{e^{-(S-S_0)t}}{S-S_0}\Big|_0^\infty=\dfrac{M}{S-S_0}$$
Так как модуль интеграла \eqref{e67} оказался при значении $S>S_0$ ограниченной величиной, то интеграл сходится и функция $F(p)$ -- существует.\\
\textit{Замечание 1}\\
Из доказанного неравенства:
\begin{equation}\label{e68}
\Big|\int\limits_0^\infty f(t)\cdot e^{-pt}dt\Big|\leqslant\dfrac{M}{S-S_0}
\end{equation}
следует, что если $p\rightarrow\infty$ так, что $Re$ $p=S\rightarrow+\infty$, то $F(p)\rightarrow0$. Действительно, $\lim\limits_{S\rightarrow\infty}\dfrac{M}{S-S_0}=0$, откуда в силу неравенства \eqref{e68} получаем требуемое.\\
\textit{Замечание 2}\\
В дальнейшем будем полагать $Re$ $p>S_0$, то есть рассматривать изображение $F(p)$ лишь для тех $p$ для которых обеспечено существование этого изображения.\\
\textit{Пример 1}\\
Пусть $f(t)-\eta(t)=\begin{cases}
0,t<0\\
1,t\geqslant0
\end{cases}$. Очевидно, что порядок роста этого оригинала $S_0=0$, его изображением будет функция: $F(p)\rightarrow\int\limits_0^\infty1\cdot e^{-pt}dt=-\dfrac{e^{-pt}}{p}\Big|_0^\infty=\dfrac{1}{p}$ при $Re$ $p>S_0=0$\\
Итак, $\eta(t)\rightarrow\dfrac{1}{p}$ или $1\rightarrow\dfrac{1}{p}$\\
\textit{Пример 2}\\
$\letus f(t)=\eta(t)\cdot e^{q_1t}=\begin{cases}
0,t<0\\
e^{q_1t},t\geqslant0
\end{cases}$, $q_1$ -- комплексное число.\\
Порядок роста оригинала $f(t)$, равен $S_0=Re$ $q_1$.\\
Его изображение: $F(p)\rightarrow\int\limits_0^\infty e^{q_1t}\cdot e^{-pt}dt=\int\limits_0^\infty e^{-(p-q_1)t}dt=\dfrac{e^{-(p-q_1)t}}{-{p-q_1}}\Big|_0^\infty=\dfrac{1}{p-q_1}$ в предположении, что $Re(p-q_1)>0$, то есть $Re$ $p>Re$ $q_1=S_0$\\
\textit{Замечание 3}\\
Условимся в дальнейшем множитель опускать и произведение функции на функцию \colorbox{red!50}{***} $\eta(t)\cdot f(t)$ обозначать $f(t)$\\
Оригинал $\eta(t)\cdot e^{-q_1t}$ будем обозначать $e^{q_1t}$, оригинал $\eta(t)\cdot\cos t$ через $\cos t$ и так далее\\
Итак, $e^{q_1t}\rightarrow\dfrac{1}{p-q_1}$
\\\\
\textbf{\large{\S3 Основные свойства преобразования Лапласа}}
\\\\
Пусть известны изображения оригиналов $f(t)$ и $\varphi(t)$\\
Как найти изображения для функций: $2f(t)$; $2f(t)-3\varphi(t)$; $f'(t)$; $\int\limits_0^\infty f(\tau)d\tau$\\
Для ответа на поставленный вопрос необходимо изучить свойства преобразования Лапласа.\\
1)Свойство линейности\\
Если $f(t)\rightarrow F(p),\varphi\rightarrow\varPhi(p)$ и $a$, $b$ -- любые постоянные, то $af(t)+b\varphi(t)\rightarrow aF(p)+b\varPhi(p)$\\
Доказательство:\\
Доказательство основано на определении \eqref{e68} преобразования Лапласа и свойстве линейности опредленного интеграла
$$af(t)+b\varphi(t)\rightarrow\int\limits_0^\infty(af(t)+b\varphi(t))e^{-pt}dt=a\int\limits_0^\infty f(t)e^{-pt}dt+b\int\limits_0^\infty\varphi(t)e^{-pt}dt=aF(p)+b\varPhi(p)$$
\textit{Пример 3}\\
$\letus f(t)=\sin t$ найти изображение этого оригинала. Эту функцию с помощью формул Эйлера можно представить в виде: $\dfrac{1}{2i}(e^{\alpha ti}-e^{-\alpha ti})$. Изображение $e^{q_1t}$ известное. Пользуясь свойством линейности, находим: $\sin\alpha t\rightarrow\dfrac{1}{2i}(\dfrac{1}{p-\alpha i}-\dfrac{1}{p+\alpha i})=\dfrac{2i\alpha}{2i(p^2+\alpha^2)}=\dfrac{\alpha}{p^2-\alpha^2}$\\
Точно также можно показать, что $\cos\alpha t\rightarrow\dfrac{p}{p^2+\alpha^2};\ch\alpha t\rightarrow\dfrac{p}{p^2-\alpha^2};\sh\alpha t\rightarrow\dfrac{\alpha}{p^2-\alpha^2}$\\
2)Свойство подобия:\\
Свойство подобия характеризует изменение масштаба вещественной переменной $t$ при преобразовании Лапласа.\\
Если $f(t)\rightarrow F(p)$, то для $\forall\omega>0:f(\omega t)\rightarrow\dfrac{1}{\omega}F(\dfrac{p}{\omega})$\\
Доказательство:
$\letus f(\omega t)\rightarrow\int\limits_0^\infty f(\omega t)\cdot e^{-pt}dt$. Заметим $\omega t=\tau$, тогда $f(\omega t)\rightarrow\dfrac{1}{\omega}\int\limits_0^\infty f(\tau)\cdot e^{-\dfrac{p}{\omega}\tau}dt=\dfrac{1}{\omega}F(\dfrac{p}{\omega})\footnote{На экзамене: где использовано условие $\omega>0$}$\\
\textit{Пример 4}\\
Найти изображение для $f(t)=\sin^2t$\\
Заменим $f(t)=sin^2t=\dfrac{1}{2}(1-\cos2t$, тогда: $\sin^2t=\dfrac{1}{2}(1-\cos2t)\rightarrow\dfrac{1}{2}\dfrac{1}{p}-\dfrac{1}{2}\dfrac{p}{p^2-4}=\dfrac{1}{2}\dfrac{p^2-4-p^2}{p(p^2-4)}=\dfrac{2}{p(p^2-4)}$\\
3)Свойство дифференцирования оригинала\\
Если $f(t)\rightarrow F(p)$ и $f'(t),f''(t),...,f^{(n)}(t)$ -- оригиналы, то\\
$f'(t)\rightarrow pF(p)-f(0)$\\
$f''(t)\rightarrow p^2F(p)-pf(0)-f'(0)$\\
$f'''(t)\rightarrow p^nF(p)-p^{n}f(0)-p^{n-1}f(0)-p^{n-2}f(0)-...-f'(0)$\\
\textit{Замечание}\\
Здесь под производной $f^{(k)}(0)$ понимается $\lim\limits_{\tau\rightarrow0}f^{(k)}(t)$\\
Доказательство: Найдем изображение для $f'(t)$: $f'(t)\rightarrow\int\limits_0^\infty f'(t)\cdot e^{-pt}dt$ Проинтегрируем интеграл \colorbox{red!50}{***} по частям: Так как предполагается, что $Re$ $p=S>S_0$, то $|e^{-pt}\cdot f(t)|\leqslant M\cdot e^{-(S-S_0)t}\rightarrow0$ при стремлении $t\rightarrow\infty$, то есть $f(t)\cdot e^{-pt}\Big|_0^\infty=0-f(0)$. Поэтому $f'(t)\rightarrow pF(p)-f(0)$\\
Найдем изображение второй производной $f''(t)=(f'(t))'$\\
Пользуясь полученным выражением для первой производной:\\
$f''(t)\rightarrow p(pF(p)-f(0))-f'(0)=p^2F(p)-pf(0)-f'(0)$ для\\
$f'''(t)\rightarrow p(p^2F(p)-pf(0)-f'(0))-f''(0)=p'F(p)-p^2f(0)-pf'(0)-f''(0)$\\
В общем случае для $n$-ой производной:\\
$f^{(n)}(t)\rightarrow p^nF(p)-p^{n-1}f(0)-p^{n-2}f(0)-...-f^{(n-1)}(0)$\\
В частном случае, если $f(0)=0$, то $f'(t)\rightarrow pF(p)$, то есть операция дифференцирования оригинала $f(t)$ сводится к умножению изображения $F(p)$ на \colorbox{red!50}{***}.\\
Если же $f(0)=f'(0)=...=f^{(n-1)}(0)=0$, то $f^n(t)\rightarrow p^nF(p)$\\
\textit{Пример 5}\\
Найти изображение для оригинала $f(t)=x'''-4x''-x'-5$, если $x(0)=1,x'(0)=0,x''(0)=-1$ и $x(t)=X(p)$\\
Решение: найдем изображение каждого слагаемого\\
$x'''\rightarrow p^3X(p)-p^2X(0)-p'(x)-x''(0)=p'X-p^2+1$\\
$x''\rightarrow p^2X(p)-pX(0)-x'(0)=p^2X-p$\\
$x'\rightarrow pX(p)-X(0)=pX-1$, $5=5\dfrac{1}{p}=\dfrac{5}{p}$\\
Тогда $f(t)=x'''-4x''-x'-5\rightarrow X(p^3-4p^2-p)-p^2+4p+2-\dfrac{5}{p}$\\
4)Свойство интегрирования оригинала\\
Если $f(t)\rightarrow F(p)$, то $\int\limits_0^\infty f(\tau)d\tau\rightarrow\dfrac{F(p)}{p}$\\
Доказательство: покажем прежде всего, что $\int\limits_0^\infty f(\tau)=\varphi(t)$ -- оригинал.\\
Дейтствительно условие \eqref{e68} определения выполнено по свойству определенного интеграла, то есть функция $f(t)$ непрерывна всюду, где непрерывна функция $f(t)$, или имеет конечно число точек разрыва 1 рода.\\
Очевидно, что $\varphi(t)=0$, при $t<0$, так как $f(t)=0$, при $t<0$
\section{Лекция 22.04.2019}
\textbf{\large{\S4 Интегрирование оригинала}}
\\\\
$|\varphi(t)|=|\int\limits_0^t\varphi(\tau)d\tau|\leqslant\int\limits_0^t|f(t)|d\tau$\\
В силу того, что $f(t)$ -- оригинал, существует $M>0$ и $S_0>0$ для которых $|f(t)|\leqslant M\cdot e^{S_0t}$, тогда и $|\varphi(t)|\leqslant M\cdot e^{S_0t}\cdot t$ или $|\varphi(t)|\leqslant M\cdot e^{S_0t}\cdot e^{\ln t}\leqslant M\cdot e^{\mbox{\colorbox{red!50}{***}}}$\\
Итак $\int\limits_0^tf(\tau)d\tau=\varphi(t)$ удовлетворяет всем требованиям, которые были наложены на функцию оригинал.\\
Найдем изображение оригинала:\\
$\letus\varphi(t)=\int\limits_0^tf(\tau)d\tau\rightarrow\varPhi(p)$\\
Очевидно, что $\varphi(0)=0$ и по свойству 3 $\varphi'(t)\rightarrow p\varPhi(p)$. Но $\varphi'(t)=f(t)$, поэтому $f(t)\rightarrow p\varPhi(p)=F(p),\varPhi(p)=\dfrac{F(p)}{p}$\\
Операция интегрирования в пространстве оригиналов соответствует операция деления в пространстве изображений.
\\\\
\textbf{\large{\S5 Дифференцирование изображений}}
\\\\
Если $F(p)\rightarrow f(t)$, то $F'(p)\rightarrow(-t)\cdot f(t)$\\
$F''(p)\rightarrow(-t)^2f(t)$\\
$F^{(n)}(p)\rightarrow(-t)^nf(t)$\\
Доказательство: по теореме о производной от интеграла по параметру: $\dfrac{d}{d\alpha}\int\limits_a^b\varphi(t,\alpha)dt=\int\limits_a^b\dfrac{\partial\varphi}{\partial\alpha}dt$ для $\forall\alpha\in[c,d]$, если $\varphi(t,\alpha),\dfrac{\partial\varphi}{\partial\alpha}$ непрерывны при $\begin{cases}
a\leqslant t\leqslant b\\
c\leqslant\alpha\leqslant d
\end{cases}$\\
По определению преобразования Лапласа $F(p)=\int\limits_0^\infty e^{-pt}\cdot f(t)dt$, а $F'(p)=\int\limits_0^\infty(-t)e^{-pt}f(t)dt$\\
Дифференцированию в пространстве изображений соответствует операция умножения оригинала на аргумент с отрицательным знаком в пространстве оригиналов.
\\\\
\textbf{\large{\S6 Интегрирование изображений}}
\\\\
Если $f(t)\rightarrow F(p),\dfrac{f(t)}{t}$ -- оригинал, а интеграл $\int\limits_0^\infty F(z)dz$ сходится, то $\dfrac{f(t)}{t}\rightarrow\int\limits_0^\infty F(z)dz$\\
Доказательство: $\dfrac{f(t)}{t}$ -- оригинал, пусть $\dfrac{f(t)}{t}\rightarrow\varPhi(p)$\\
По свойству 5: $f(t)\rightarrow-\varPhi'(p)$. Но $f(t)\rightarrow F(p)$, поэтому $F(p)=-\varPhi'(p)$ или $d\varPhi=-F(p)dp$\\
Имеется Д.У. с разделяющимися переменными, проанализируем его в пределах от значения $p$ до значения $\eta$: $\int d\varPhi=\varPhi(\eta)-\varPhi(p)=-\int\limits_p^\eta F(z)dz$. Положим, $\eta\rightarrow\infty$, тогда $\lim\limits_{\eta\rightarrow\infty}\varPhi(p)=0,\lim\limits_{\eta\rightarrow\infty}\int\limits_p^\eta F(z)dz=\int\limits_p^\infty F(z)dz$ Следовательно: $\varPhi(p)=\int\limits_p^\infty F(z)dz$, но $\dfrac{f(t)}{t}\rightarrow\varPhi(p)$, поэтому: $\dfrac{f(t)}{t}\rightarrow\int\limits_p^\infty F(z)dz$. Это соответствие отсутствует, если $Re$ $p>S_1>S_0$, где $S_0$ и $S_1$ -- показатели роста функции $f(t)$ и $\dfrac{f(t)}{t}$ соответственно.\\
Операция деления на аргумент в пространстве оригинала соответствует операция интегрирования в пределах от $p$ до $\alpha$ в пределах пространства изображения.
\\\\
\textbf{\large{\S7 Смещение в аргументе изображения}}
\\\\
Если $f(t)\rightarrow F(p)$ и $\alpha$ -- комплексное число, то $e^{-\alpha t}f(t)\rightarrow F(p+\alpha)$\\
Доказательство: применим преобразование Лапласа к оригиналу $e^{-\alpha t}f(t):e^{-\alpha t}f(t)\rightarrow\int\limits_0^\infty e^{-\alpha t}f(t)\cdot e^{-pt}dt=\int\limits_0^\infty f(t)\cdot e^{-(p+\alpha)t}dt=F(p+\alpha)$\\
Показатель роста оригинала $f(t)-S_0$, а показатель роста $e^{-\alpha t}f(t)$ будет $(S_0-Re$ $\alpha$. В связи с этим, утверждение 7 справедливо, если $Re$ $p>(S_0-Re$ $\alpha$ или $Re(p+\alpha)>S_0$
\\\\
\textbf{\large{\S8 Смещение в аргументе оригинала (запаздывание)}}
\\\\
Если $f(t)\rightarrow F(p)$ и $f(t-a)=0$ при значениях $t<a$, то для всего $a>0:f(t-a)\rightarrow e^{-pa}F(p)$\\
Иначе говоря, если процесс описываемый не оригиналом $f(t-a)$ опаздывает на время $a$ по сравнению с первоначальным $f(t)$, то изображение, соответствующее этому процессу, получается из изображения первоначального оригинала умножением на функцию $e^{-pa}$\\
Доказательство: используем определение преобразование Лапласа для оригинала $f(t-a)$ и свойство аддитивности определенного интеграла относительно отрезка интегрирования:\\ $$f(t-a)\rightarrow\int\limits_0^a\underbrace{f(t-a)e^{-pt}}_{\begin{matrix}
=0\\
f(t-a)=0\\
\mbox{при }t<0\mbox{ по}\\
\mbox{усл-ю теоремы}
\end{matrix}}dt+\underbrace{\int\limits_0^\infty f(t-a)\cdot e^{-pt}dt}_{\begin{matrix}
\mbox{применяем замену}\\
\mbox{переменной }\Big|\begin{matrix}
t-a=\tau\\
dt=d\tau
\end{matrix}\Big|\\
\mbox{при значении}\\
t=a\Rightarrow\tau=0\\
t=\infty\Rightarrow\tau=\infty
\end{matrix}}$$
Тогда $f(t-a)\rightarrow\int\limits_0^\infty f(\tau)\cdot e^{-p(a-\tau)}d\tau=\int\limits_0^\infty f(\tau)\cdot e^{-pa}\cdot e^{-p\tau}d\tau=e^{-pa}\cdot\int\limits_0^\infty f(\tau)\cdot e^{-p\tau}d\tau=e^{-pa}:F(p)$\\
Графически:\\\\
\begin{tikzpicture}
\draw[->] (-1,0)--(4,0) node [right] {t};
\draw[->] (0,-1)--(0,2) node [above] {f(t)};
\node at (0,0) [below left] {0};
\node at (1,0) [below] {a};
\draw[dashed] (0,1) .. controls (2,2) and (2,0) .. (4,1);
\draw (1,1) .. controls (3,2) and (3,0) .. (5,1) node [right] {$f(t-a)$};
\draw[dashed] (1,1)--(0,1);
\draw[dashed] (1,1)--(1,0);
\end{tikzpicture}\\
\\\\
\textbf{\large{\S9 Изображение периодического оригинала}}
\\\\
Если $f(t)\rightarrow F(p)$ и $f(t)$ -- периодическая функция с периодом $\tau>0$, то $F(p)=\dfrac{1}{e^{-pT}}\int\limits_0^Tf(t)\cdot e^{-pt}dt$\\
Доказательство: По определению преобразования Лапласа:\\
$F(p)=\int\limits_0^\infty f(t)\cdot e^{-pt}dt=\int\limits_0^Tf(t)\cdot e^{-pt}dt+\int\limits_T^\infty e^{-pt}dt$\\
Применяем по второму интегралу замену переменной $t=\tau+T\rightarrow dt=d\tau$\\
$t=T\Rightarrow\tau=0$\\
$t\rightarrow\infty\Rightarrow\tau=\infty$\\
$F(p)=\int\limits_0^Tf(t)\cdot e^{-pt}dt+\int\limits_0^\infty f(\tau)\cdot e^{-p\tau}\cdot e^{-pT}d\tau=\int\limits_0^Tf(t)\cdot e^{-pt}dt+e^{-pT}F(p)$\\
А от сюда, $F(p)=\dfrac{1}{1-e^{-pt}}\int\limits_0^Tf(t)\cdot e^{-pt}dt$, что и требовалось доказать.
\\\\
\textbf{\large{\S10 Свертка функций. Теорема умножения}}
\\\\
$\letus f(t)$ и $\varphi(t)$ -- непрерывны для значений $t>0$. Сверткой этих двух двух функций называется интеграл $f*\varphi$ или $f\cdot\varphi$\\
$f*\varphi=\int\limits_0^t\varphi(\tau)f(t-\tau)d\tau=\Big|\begin{matrix}
t-\tau=u\\
-d\tau=du
\end{matrix}\Big|=-\int\limits_\tau^0\varphi(t-u)\cdot f(u)du=\int\limits_0^tf(u)\varphi(t-u)du=f*\varphi$\\
Можно показать, что если $f(t)$ и $\varphi(t)$ -- оригиналы, то их свертка тоже является оригиналом, а точнее, если $f(t)$ и $\varphi(t)$ -- оригиналы с показателем роста $S_0$ и $S_1(S_0\rightarrow S_1)$, то $f*\varphi$ -- оригинал с показателем роста $S_0$.\\
Рассмотрим часто встречающийся случай, когда изображение неизвестного оригинала разлагается на множители.\\
$F(p)\cdot\varPhi(p)$ причем известны оригиналы соответствующие множителям $F(p)\rightarrow f(t), \varPhi(p)\rightarrow\varphi(t)$\\
Можно ли найти неизвестный оригинал?\\
\textit{Теорема}\\
Если $f(t)\rightarrow F(p),\varphi(t)\rightarrow\varPhi(p)$, то произведение изображений $F(p)\cdot\varPhi(p)\rightarrow f(t)*\varphi(t)$\\
Иначе говоря умножение изображений равносильно свертыванию оригиналов этих изображений.\\
Доказательство:\\
По определению изображения $f(t)*\varphi(t)\rightarrow\int\limits_0^\infty f*\varphi\cdot e^{-pt}dt=\int\limits_0^\infty e^{-pt}dt\int\limits_0^t f(t-\tau)*\varphi(\tau)d\tau$\\
Как известно, интеграл Лапласа абсолютно сходится при значениях $Re$ $p>S_0$, поэтому можно изменить порядок интегрирования $f*\varphi\rightarrow\int\limits_0^\infty\varphi(\tau)d\tau\int\limits_\tau^\infty e^{-pt}f(t-\tau)d\tau=\bigg|\begin{matrix}
t-\tau=u\\
dt=du
\end{matrix}\bigg|=\int\limits_0^\infty\varphi(\tau)d\tau\cdot e^{-pt}\int\limits_0^\infty f(u)\cdot e^{-pu}du=\int\limits_0^\infty\varphi(\tau)e^{-p\tau}\cdot\int\limits_0^\infty f(u)\cdot e^{-pu}du=\varPhi(p)\cdot F(p)$
\\\\
\textbf{\large{\S11 Интеграл Дюамеля}} (Жан-Мари Констан Дюамель, 1791--1872)
\\\\
Формула является следствием теоремы умножения и имеет применение при расчете переходных процессов в электрических цепях.\\
Если $f(t)\rightarrow F(p),\varphi(t)\rightarrow\varPhi(p)$, то $p\cdot F\cdot\varPhi\rightarrow f(t)\varphi(0)+\underbrace{\int\limits_0^tf(\tau)\cdot\varphi'(t-\tau)d\tau}_{\mbox{Интеграл Дюамеля}}$\\
Эта формула может быть использована для решения Л.Н.Д.У. с постоянными коэффициентами при нулевых начальных условиях. А правая часть меняется неоднократно или когда для правой части трудно подобрать изображение.\\
Нахождение оригинала по изображению:\\
Теорема единственности\\
Каждому оригиналу $f(t)$ соответствует единственное изображение $F(p)$ обратные не всегда верно, то есть одна и та же функция может служить изображением различных оригиналов. Таким образом по скольку такое изменение не влияет на результат изменения вычисления интеграла Лапласа изображение не изменится.\\
\textit{Теорема единственности}\\
Если функция $F(p)$ является изображением оригиналов $f_1(t)$ и $f_2(t)$, то эти оригиналы равны во всех точках $t$, где функции $f_1(t)$ и $f_2(t)$ непрерывны.\\
Из теоремы следует, что если $F(p)$ -- изображение непрерывного оригинала $f(t)$, то этот оригинал единственный. Так как в приложениях оригиналов предполагается дифференцируемыми функциями, то обычно находят изображение оригинала по таблице соответствия.\\
$F(p)=\dfrac{p}{p^2-4}\rightarrow\cos2t$ -- единственный непрерывный при $t>0$ оригинал.
\section{Лекция 29.04.2019}
Таблица соответствия:\\
Когда изображение отсутствует, то это изображение стремятся выразить через линейную комбинацию или несколько выражений.\\
\textit{Пример}\\
Найти оригинал для выражения: $F(p)=\dfrac{4p-3}{p^2-4p+3}$\\
Решение: разложить дробь на элекментарные и воспользоваться таблицей соответствия.\\
$F(p)=\dfrac{4p-3}{p^2-4p+3}=-\dfrac{1}{2}\dfrac{1}{p-1}+\dfrac{9}{2}\dfrac{1}{p-3}\rightarrow-\dfrac{1}{2}e^t+\dfrac{9}{2}e^3t=f(t)$\\
Приложение операционного исчисления:\\
Решение Л.Д.У. второго порядка с постоянными коэффициентами\\
Основные этапы реализации операционного метода\\
1) искомой функции $f(t)$ ставят в соответствие другую функцию $F(p)$ -- изображение функции $f(t)$\\
2) Над функцией $F(p)$ проводят операции соответствующих заданным операциям над функцией $f(t)$ и получают вспомогательные уравнения относительно $F(p)$\\
3) Последнее уравнение разрешают относительно функции $F(p)$, что обычно значительно проще, чем нахождение $f(t)$ из исходного уравнения\\
4) По решению $F(p)$ вспомогательного уравнения находят функцию $f(t)$, которая является искомой\\
Рассмотрим последовательность выполнения этих этапов:\\
1) Дано неоднородное Л.Д.У. второго порядка с постоянными коэффициентами:
\begin{equation}\label{e69}
x''+a_1x'+a_2x=f(t)    
\end{equation}
Требуется найти его частное решение удовлетворяющее Н.У.:
\begin{equation}\label{e70}
x(0)=x_0, x'(0)=x_0'
\end{equation}
Будем считать, что искомым решением является $x(t)$, причем, его производные $x',x'',a$ также функция $f(t)$ -- оригиналы. Введем в рассмотрение новые функции: $X(p)\rightarrow x(t),F(p)\rightarrow f(t)$, тогда $x'(t)\rightarrow pX(p)-x_0, x''(t)\rightarrow p^2X(p)\rightarrow p^2X(p)-px_0-x_0'$\\
Используя теорему линейности и единственности изображения перейдем в \eqref{e69} от оригиналов к изображениям
\begin{equation}\label{e71}
p^2X-px_0x_0'+a_1(pX-x_0)+a_2X=F(p)
\end{equation}
Уравнение \eqref{e71} -- вспомогательно уравнение или уравнение в изображениях, соответствующее Д.У. \eqref{e69} при Н.У. \eqref{e70}. Следовательно решение Д.У. относительно оригинала $x(t)$ сводится к решению линейного алгебраического уравнения относительно изображения $X(p)$.\\
$X(p)=\dfrac{px_0+x_0'+a_1x_0+F(p)}{p^2+a_1p+a_2}$\\
Полагая $x(t),x'(t),x''(t)$ оригиналами, мы тем самым условились, что нас интересует решение уравнения \eqref{e69} при $t\geqslant0$. При решении конкретных задач получившееся решение часто оказывается справедливым и при $t<0$, но это требует дополнительной проверки.\\
\textit{Пример}\\
Найти решение уравнения $x''-x'=2$\\
Н.У. $x(0)=1,x'(0)=-1$\\
Решение: обозначим $X(p)\rightarrow x(t)\Rightarrow$ по правилу дифференцирования оригинала: $x'(t)\Rightarrow pX-1,x''(t)\rightarrow p^2X-p+1\Rightarrow$ вспомогательное или операторное уравнение выглядит так:
$p^2X-p+1-pX+1=\dfrac{2}{p}$\\
$X(p^2-p)=\dfrac{2}{p}+p-2$\\
Отсюда: $X=\dfrac{-2(p^2-p)+p^2}{p^2(p^2-1)}=-\dfrac{2}{p}+\dfrac{1}{p-1}\rightarrow-2t+e^t=x(t)$ при $t\geqslant0$ решение исходного Д.У.
\\\\
\textbf{\Large{Глава 4 Ряды}}
\fancyhead[R]{Ряды}
\\\\
\textbf{\large{\S1 Бесконечный ряд. Его сходимость}}
\\\\
\textbf{1. Основные понятия}
\\\\
$\letus$ задано $\infty$ последовательность чисел: $u_1,u_2,...,u_n,...$\\
Числовым рядом называется составленные из этих чисел выражение: $u_1+u_2+...+u_n+...=\sum\limits_{n=1}^\infty u_n$\\
Числа $u_1,u_2,...$ -- называются членами ряда, $u_n$ -- общим членом ряда\\
Конечная сумма $S_n=u_1+u_2+...+u_n$ называется $n$-ой частичной суммой ряда.\\
Если $\exists$ конечный предел $\lim\limits_{n\rightarrow\infty}S_n$, ряд называется сходящимся.\\
Если ряд сходится, то число $S=\lim\limits_{n\rightarrow\infty}S_n$ называется суммой ряда, а разность $Z_n=S-S_n=u_{n+1}+u+_{n+2}+...$ -- называется остатком ряда после $n$-ого числа.\\
\textit{Пример}\\
Показать, что гармонический ряд расходится\\
$1+\dfrac{1}{2}+\dfrac{1}{3}+...+\dfrac{1}{n}\footnote{общий член}+...=\sum\limits_{n=1}^\infty\dfrac{1}{n}$\\
Так как каждый его член (кроме первого) является средним гармоническим двух членов соседних с то есть $\dfrac{1}{n}=\dfrac{1}{2}(\dfrac{1}{u_{n-1}}+\dfrac{1}{u_{n+1}})$\\
Решение: для гармонического ряда:\\
$S_n=1+\dfrac{1}{2}+\dfrac{1}{3}+...+\dfrac{1}{n}$\\
$S_{2n}=1+\dfrac{1}{2}+\dfrac{1}{3}+...+\dfrac{1}{n}+\dfrac{1}{n+1}+...+\dfrac{1}{2n}$\\
$S_{2n}=\dfrac{1}{n}+\dfrac{1}{n+1}+...+\dfrac{1}{2n}>n\cdot\dfrac{1}{2n}=\dfrac{1}{2}$\\
или: $S_{2n}>S_n+\dfrac{1}{2}$\\
Если бы гармонический ряд сходится, то последовательность $S$, его частных сумм имела бы конечный предел $A$. $\lim\limits_{n\rightarrow\infty}S_n=A$\\
И тогда и ее подпоследовательность $S_{2n}'$ имела бы тот же предел $\lim\limits_{n\rightarrow\infty}S_{2n}'=A$\\
А в предшествующем неравенстве был бы возможен предельный переход, который привел бы к соотношению $A\geqslant A+\dfrac{1}{2}\Rightarrow$ противоречие $\Rightarrow$ числовой гармонический ряд является расходящимся.
\\\\
\textbf{2. Необходимый признак сходимости}
\\\\
Если ряд сходится, то его общий член $u_n$ стремится к 0 при $n\rightarrow\infty$. Отсюда следует, что если $s_n$ не стремится к 0, то ряд расходится. Указанный признак не является достаточным, то есть если $u_n\rightarrow0$, то о сходимости ряда ничего еще сказать нельзя: он может быть как сходящимся, так и расходящимся.
\\\\
\textbf{3. Исследование на сходимость рядов с положительными членами. Признаки сравнения.}
\\\\
Достаточные признаки\\
Признаки сравнения\\
Если даны два ряда $\sum u_n$ и $\sum v_n$ с неотрицательными членами, причем члены первого ряда непревосходят соответствующих членов второго ряда: $0\leqslant u_n\leqslant v_n$, то\\
1) Из сходимости второго ряда следует сходимость первого ряда.\\
2) Из расходимости первого ряда $\Rightarrow$ расходимость второго ряда.\\
Для сравнения часто используют ряды:\\
1. $\sum\limits_{n=1}^\infty aq_1^{n-1},|q_1|<1$ -- сходится (геометрическая прогрессия)\\
2. $\sum\limits_{n=1}^\infty\dfrac{1}{n^\alpha}\begin{cases}
\mbox{сходящийся при }\alpha>1\\
\mbox{расходящийся при }\alpha\leqslant1
\end{cases}$\\
В случае $\alpha=1$ имеем гармонический ряд\\
2. Предельная форма признака сравнения:\\
Если $\sum u_n$ и $\sum v_n$ ряды с положительными членами и существует конечный: $\lim\limits_{n\rightarrow\infty}\dfrac{u_n}{v_n}=k>0$, то рассматриваемые ряды одновременно сходятся или расходятся.\\
3. Признак Коши:\\
Если для ряда $\sum u_n$ с положительными элементами существует $\lim\limits_{n\rightarrow\infty}\sqrt[n]{u_n}=l$, то этот ряд сходится при $l<1$ и расходится при $l>1$, а при $l=1$, вопрос остается открытым.\\
4. Признак Даламбера\\
Если ряд $u_1+u_2+...+u_n+...$ с положительными членами таков, что существует $\lim\limits_{n\rightarrow\infty}\dfrac{u_{n+1}}{u_n}=l$, то при $l<1$ ряд сходится, $l>1$ расходится, $l=1$ не дает ответа на вопрос.\\
Абсолютная сходимость\\
Теорема Лейбница о сходимости знакочередующихся рядов $\sum\limits_{n=1}^\infty(-1)^n\dfrac{1}{n}$\\
В теореме рассматриваются ряды с членами имеющими любой знак.\\
Если ряд $\sum u_n$ сходится, а ряд $\sum|u_n|$ расходится, то $\sum u_n$ называется условно сходящимся.\\
Если ряд $\sum|u_n|$ сходится, сходится и $\sum u_n$ - абсолютно сходящийся.\\
Ряд $a_1-a_2+a_3-a_4+a_5-...+(-1)^{n+1}a_n+...$ где все $a_n>0$ называется знакочередующимся.\\
\textit{Теорема Лейбница}\\
Если члены знакочередующегося ряда удовлетворяют условиям:\\
1) $a_1>a_2>...>a_n>...$\\
2) $\lim\limits_{n\rightarrow\infty}a_n=0$\\
то такой ряд сходится.\\
Ряд удовлетворяющий указанным условиям называется рядом Лейбница.
\section{Лекция 06.05.2019}
\textbf{Функциональные ряды. Область сходимости.}
\\\\
$\letus$ задана последовательность функций: $u_1(x),u_2(x),...,u_n(x),...$ имеющих общую область определения.\\
Функциональным рядом называется составление из этих функций выражения: $u_1(x)+u_2(x)+...+u_n(x)+...=\sum\limits_{n=1}^\infty u_n(x)$\\
Для каждого значения $x_0$ из этой области, функциональный ряд обращается в числовой ряд: $u_1(x_0)+u_2(x_0)+...+u_n(x_0)=\sum\limits^\infty$\\
Если последний член ряда сходится, то $x_0$ называется точкой сходимости функционального ряда. Множитель всех точек сходимости называется областью сходимости функционального ряда. Область сходимости находится с помощью рассмотрения признака сходимости.\\
\textit{Пример}\\
Найти область сходимости функционального ряда\\
$e^{-x}+e^{-2x}+...+e^{-nx}+...=\sum\limits_{n=1}^\infty e^{-nx}$\\
$e^{-1}+e^{-2}+...+e^{-n}+...$
\\\\
\textbf{Степенные ряды. Область сходимости.}
\\\\
Степенным рядом называется ряд вида: $a_0+a_1(x-a)+a_2(x-a)^2+...+a_n(x-a)^n+...=\sum\limits_{n=0}^\infty a_n(x-a)^n$\\
где $a_n$ - числа, называются коэффициентами степенного ряда (некоторые из них могут быть нулевыми)\\
При $a=0$, степенной ряд имеет вид: $a_0+a_1x+a_2x+...=\sum\limits a_nx$\\
Этот ряд всегда сход при $x=0$. Если же он сходится в точке $x\neq0$, то существует число $R>0$, такое что при всех $|x|<R$, степенной ряд сходится, при всех $|x|<R$, степенной ряд расходится. Это число называется $R$ называется радиусом сходимости, интервал $(-R,R)$ -- интервалом \colorbox{red!50}{***}. Если степенной ряд сходится на всей числовой оси, полагают $R=\infty$, если же он сходится только при $x=0$, \colorbox{red!50}{***} $R=0$. На концах интервала сходимости, степенной ряд может как сходится, так и расходится. Внутри интервала -- абсолютно всегда сходится. \colorbox{red!50}{***} Деламбера и Коши.
\\\\
\textbf{Ряд Тейлора}
\\\\
Если функция $f(x)$ имеет на некотором интервале, содержит точку $a$, производные всех \colorbox{red!50}{***}, то к ней может быть применена формула Тейлора:\\
$f(x)=f(a)+f'(a)(x-a)+\dfrac{f''(a)}{2!}(x-a)^2+...+\dfrac{f^{(n-1)}(a)}{(n-1)!}(x-a)^{n-1}+R_n$, где $R_n=\dfrac{f^{(n)}(\xi)}{n!}(x-a)^{n}$\\
$\xi$ -- между $a$ и $x$, $n$ -- люое натуральное число.\\
Если для некоторого значения $x$ этот член $R_n\rightarrow0$, при $n\rightarrow\infty$, то в пределе функция Тейлора превращается для этого значения $x$ в ряд Тейлора:\\
$f(x)=f(a)+f'(a)(x-a)+...+\dfrac{f^{(n)}(a)}{n!}(x-a)^n+...=\sum\limits\dfrac{f^{(n)}(a)}{n!}(x-a)^n$\\
В точке $a=0$ ряд Тейлора -- ряд Маклорена\\
$f(x)=f(0)+f'(0)x+...+\dfrac{f^{(n)}(0)}{n!}x^n+...=\sum\dfrac{f^{(n)}(0)}{n!}x^n$\\
Таким образом функция $f(x)$ может быть разложен в ряд Тейлора для расходящегося значения $x$:\\
1) она имеет производную всех порядков\\
2) остаточный член $a_n\rightarrow0$, при $n\rightarrow\infty$, для расходящегося значения $x$\\
Применение табличных \colorbox{red!50}{***} разложений.\\
Для определения коэффициента разложения функции в степенной ряд Тейлора прибегая к многократному дифференцированию и нахождению значений производной в данной точке. Затем изучают для каких значений $x$, остаточный член \colorbox{red!50}{***} при $n\rightarrow\infty$\\
Часто операция дифференцирования \colorbox{red!50}{***} технически трудными выборками, а исследование \colorbox{red!50}{***}\\
Эти трудности иногда можно обойти на основании теоремы единственности разложения функции в степенной ряд:\\
\colorbox{red!50}{***} разложения функции в степенной ряд будет ее разложением в ряд Тейлора\\
Рассмотрим последовательность выполнения этих этапов\\
1) Дано Н.Л.Д.У. второго порядка с постоянными коэффициентами
\begin{equation}\label{e720000000000000000000000000000000000000000000000000}
x''+a_1x'+a_2x=f(t)
\end{equation}
\\\\-----------------------------------------------------------------------------------\colorbox{red!50}{***}-----------------------------------------------------------------------------------
\section{Лекция 20.05.2019}
\textbf{\large{\S1 Ряды Фурье}}
\\\\
\textbf{1. Разложение в ряд Фурье с периодом $2\pi$}
\\\\
Пусть $f(x)$ -- какая-то функция с периодом $2\pi$. Рядом Фурье тригнометрической функции $f(x)$ называется тригонометрический ряд
\begin{equation}\label{e72}
f(x)=\dfrac{a_0}{2}+\sum\limits_{n=1}^{+\infty}(a_n\cos nx+b_n\sin nx)
\end{equation}
коэффициенты которого определяются строго по формулам Фурье\\
$a_0=\dfrac{1}{n}\int\limits_{-\pi}^{\pi}f(x)dx$\\
$a_n=\dfrac{1}{\pi}\int\limits_{-\pi}^{\pi}f(x)\cos nxdx,n=1,2,3,...$\\
$b_n=\dfrac{1}{\pi}\underbrace{\int\limits_{-\pi}^{\pi}f(x)\sin nxdx}_{\mbox{интеграл Фурье}},n=1,2,3,...$\\
\textit{Определение}\\
Функция $f(x)$ при этом предполагается такой, что интеграл Фурье имеет смысл, то есть $f(x)$ -- непрерывна или ограничена с конечным числом разрывов на интервале. Из этого определения не следует, что $f(x)$ всегда разлагается в ряд Фурье, то есть является его суммой \colorbox{red!50}{даже если ряд сходится}.\\
Если ряд \eqref{e72} является рядом Фурье
\begin{equation}\label{e73}
f(x)\approx\dfrac{a_0}{2}+\sum\limits_{n=1}^\infty(a_n\cos nx+b_n\sin nx)
\end{equation}
Знак соответствия ($\approx$) в \eqref{e73} можно заменить на знак равенства, если функция $f(x)$ удовлетворяет следующим условиям:\\
1) Если функция $f(x)$  имеет период $2\pi$, причем функция $f(x)$ и ее производная в сумме непрерывна или допускает разрыв I рода, в конечном числе на каждом периоде, то ряд Фурье для $f(x)$ сходится для всех $x$, а его сумма равна $f(x)$. В каждой точке непрерывности этой функции и равна числу $\dfrac{f(x-0)+f(x+0)}{2}$ в каждой точке разрыва.\\
Существуют и иные достаточные признаки разложения ряда Фурье.
\\\\
\textbf{2. Разложение в ряд Фурье четных и нечетных}
\\\\
Ряд Фурье, четной функции, то есть удовлетворяющей условию: $f(-x)=f(x)$, не содержит членов с $\sin$.\\
Этот ряд имеет вид:\\
$f(x)\approx\dfrac{a_0}{2}+\sum\limits_{n=1}^\infty a_n\cos nx$, где $a_0=\dfrac{2}{\pi}\int\limits_0^\pi f(x)\cos ndx,a_n=\dfrac{2}{\pi}\int\limits_0^\pi f(x)\cos ndx,n=1,2,3,...$\\
Ряд Фурье нечетной функции, то есть удовлетворяющей условию $f(-x)=-f(x)$ не содержит свободного члена и членов, содержащих косинусы. Этот ряд таков:\\
$f(x)\approx\sum\limits_{n=1}^\infty b_n\sin nx$, где $b_n=\dfrac{2}{\pi}\int\limits_0^\pi f(x)\sin ndx,n=1,2,3,...$
\\\\
\textbf{3. Разложение в ряд Фурье функций, заданных на полупериоде от 0 до $\pi$}
\\\\
Функцию заданную на  полупериоде $(0,\pi)$ можно разложить (по желанию) в ряд синусов и в ряд косинусов, продолжая на втором полупериоде $(-\pi,0)$ соответственно нечетным или четным образом.\\
\textit{Пример}\\
Функцию $f(x)=\dfrac{\pi}{4}-\dfrac{x}{2}$ разложить в ряд косинусов на интервале $(0,\pi)$\\\\
\begin{tikzpicture}
\draw[->] (-3,0)--(3,0);
\draw[->] (0,-3)--(0,3);
\draw[thick] (0,1)--(2,0);
\draw[thick] (0,1)--(-2,0);
\draw[thick] (0,-1)--(-2,0);
\node at (2,0) [below] {$\pi$};
\node at (0,0) [below left] {0};
\node at (-1,-1) {нечет};
\node at (-1,1) {чет};
\end{tikzpicture}\\\\
Решение: продолжая эту функцию четным образом, как показано на рисунке пунктиром имеем:\\
$a_0=\dfrac{2}{\pi}\int\limits_0^\pi(\pi-\dfrac{x}{2})dx=\dfrac{2}{\pi}(\dfrac{\pi}{4}x-\dfrac{x^2}{4})\bigg|_0^\pi=\dfrac{2}{\pi}(\dfrac{\pi^2}{4}-\dfrac{\pi^2}{4})=0$\\
$a_n=\dfrac{2}{\pi}\int\limits_0^\pi(\dfrac{\pi}{4}-\dfrac{x}{2})\cos ndx=(\dfrac{\pi}{4}-\dfrac{x}{2})\sin\dfrac{nx}{x}\bigg|_0^\pi+\dfrac{1}{\pi n}\int\limits_0^\pi\sin nxdx=-\dfrac{\cos nx}{\pi n^2}\bigg|_0^\pi=\dfrac{1}{n^2\pi}-(-1)^n\dfrac{1}{\pi n^2}=\dfrac{1}{\pi n^2}(1-(-1)^n)$\\
$a_{2k}=0$ -- для четного $n$, $a_{2k+1}=\dfrac{2}{\pi(2k+1)^2}$ -- для нечетного $n$\\
\\\\
\textbf{4. Разложение в ряд Фурье функций с полупериодом $2l$}
\\\\
$l$ -- произвольное число.\\
Для функции с любым периодом $2l$ разложение в ряд Фурье, когда оно возможно \colorbox{red!50}{***} формулы коэффициентов Фурье таковы:\\
$f(x)=\dfrac{a_0}{2}+\sum\limits_{n=1}^\infty(a_n\cos\dfrac{n\pi x}{l}+b_n\sin\dfrac{n\pi x}{l}$, $a_0=\dfrac{1}{l}\int\limits_{-l}^{l}f(x)dx$, $a_n=\dfrac{1}{l}\int\limits_{-l}^{l}f(x)\cos\dfrac{n\pi x}{l}dx$, 
$b_n=\dfrac{1}{l}\int\limits_{-l}^{l}f(x)\sin\dfrac{n\pi x}{l}dx$\\
В частности, если функция $f(x)$ -- четная, то $f(x)=\dfrac{a_0}{2}+\sum\limits_{n=1}^\infty a_n\cos\dfrac{n\pi x}{l}$, $a_0=\dfrac{2}{l}\int\limits_0^lf(x)dx$\\ $a_n=\dfrac{2}{l}\int\limits_0^lf(x)\cos\dfrac{n\pi x}{l}dx$\\
Если функция $f(x)$ -- нечетная, то $f(x)=\sum\limits_{n=1}^\infty b_n\sin\dfrac{\pi nx}{l}$, $b_n=\dfrac{2}{l}\int_0^\infty f(x)\sin\dfrac{\pi nx}{l}dx$\\
\textit{Пример}\\
Разложить в ряд Фурье функцию $f(x)$ с периодом 4\\\\
\begin{tikzpicture}
\draw[->] (-2.5,0)--(2.5,0);
\draw[->] (0,-2.5)--(0,2.5);
\draw[thick,red] (-1,-1)--(1,1);
\draw[thick,red] (2,0)--(1,0);
\draw[thick,red] (-2,0)--(-1,0);
\node at (-2,0) [below] {-2};
\node at (-1,0) [below] {-1};
\node at (1,0) [below] {1};
\node at (2,0) [below] {2};
\node at (0,1) [left] {1};
\node at (0,-1) [left] {-1};
\end{tikzpicture}\\\\
Заданная функция нечетная с периодом $2l=4$, поэтому\\
$b_n=\int\limits_0^2f(x)\sin\dfrac{n\pi x}{2}dx=\int\limits_0^1x\sin\dfrac{n\pi x}{2}dx=-\dfrac{2}{n\pi}x\cos\dfrac{n\pi x}{2}\bigg|_0^1+\dfrac{2}{n\pi}\int\limits_0^1\cos\dfrac{n\pi x}{2}dx=-\dfrac{2}{n\pi}\cos\dfrac{n\pi}{2}+\dfrac{4}{n\pi^2}\sin\dfrac{n\pi x}{2}\bigg|_0^1=-\dfrac{2}{n\pi}\cos\dfrac{n\pi}{2}+\dfrac{4}{n^2\pi^2}\sin\dfrac{n\pi}{2}=\begin{cases}
\dfrac{(-1)^{k+1}}{k\pi},\mbox{ при }n=2k\\
\dfrac{(-1)^k}{(2k+1)^2}\cdot\dfrac{4}{\pi^2},\mbox{ при }n=2k+1
\end{cases}$\\
Очевидно, в силу разложенности $f(x)$ в ряд Фурье, получим:\\
$f(x)=\dfrac{4}{\pi^2}\sin\dfrac{\pi x}{2}+\dfrac{1}{\pi}\sin\pi x-\dfrac{4}{3^2\pi^2}\sin\dfrac{3\pi x}{2}-\dfrac{1}{25}\sin\pi x+...$
\\\\
\textbf{5. Разложение в ряд Фурье функций с двойной симметрией}
\\\\
Если функция $f(x)$ периода $2l$ четная или нечетная и удовлетворяет условию: $f(l-x)=f(x)$, то говорят, что $f(x)$ обладает двойной симметрией: В этом случае, если $f(x)$ -- четная, ее коэффициеты Фурье приобретают следующий вид $a_{2n}=\dfrac{4}{l}\int\limits_0^{l/2}f(x)\cos\dfrac{2\pi nx}{l}dx$, $a_{2n+1}=0(n=0,1,2,...)$, $b_n=0(n=0,1,2,...)$\\
Если $f(x)$ -- нечетная функция, то коэффициенты разложения имеют такой вид: $a_n=0(n=0,1,2,...)$, $b_{2n}=0(n=0,1,2,...)$, $b_{2n+1}=\dfrac{4}{l}\int\limits_0^{l/2}f(x)\sin\dfrac{\pi(2n+1)}{l}dx,n=0,1,2,...$
\newpage
\begin{flushright}
ПРИЛОЖЕНИЕ А
\fancyhead[R]{ПРИЛОЖЕНИЕ А}
\end{flushright}
\begin{center}
Список сокращений
\end{center}
Д.У. -- дифференциальное уравнение\\
Л.Д.У. -- линейное дифференциальное уравнение\\
Л.О.Д.У. -- линейное однородное дифференциальное уравнение\\
Н.Л.Д.У -- неоднородное линейное дифференциальное уравнение\\
Н.У. -- начальные условия\\
О.Д.У. -- однородное дифференциальное уравнение\\
О.И. -- операционное исчисление\\
О.Р. -- общее решение\\
О.Л.Д.У. -- однородное линейное дифференциальное уравнение\\
О.Р.Д.У. -- общее решение дифференциального уравнения\\
Ф.С.Р. -- фундаментальная система решений\\
\newpage
\begin{flushright}
ПРИЛОЖЕНИЕ Б
\fancyhead[R]{ПРИЛОЖЕНИЕ Б}
\end{flushright}
\begin{center}
Таблица соответствий оригиналов и изображений\\\vspace*{0.5cm}
\begin{tabular}{|c|c|}
\hline
Оригинал & Изображение \\ \hline
$\eta(t)=1$ & $\dfrac{1}{p}$ \\ \hline
$e^{\alpha t}$ & $\dfrac{1}{p-\alpha}$ \\ \hline
$\sin\omega t$ & $\dfrac{\omega}{p^2+\omega^2}$ \\ \hline
$\sin t$ & $\dfrac{1}{p^2+1}$ \\ \hline
$\cos\omega t$ & $\dfrac{\omega}{p^2+\omega^2}$ \\ \hline
$\cos t$ & $\dfrac{1}{p^2+1}$ \\ \hline
$\sh\omega t$ & $\dfrac{\omega}{p^2-\omega^2}$ \\ \hline
$\ch\omega t$ & $\dfrac{p}{p^2-\omega^2}$ \\ \hline
$e^{\alpha t}\sin\omega t$ & $\dfrac{\omega}{(p-\alpha)^2+\omega^2}$ \\ \hline
$e^{\alpha t}\cos\omega t$ & $\dfrac{\omega}{(p-\alpha)^2+\omega^2}$ \\ \hline
$t^n$ & $\dfrac{n!}{p^{n+1}}$ \\ \hline
$t^ne^{\alpha t}$ & $\dfrac{n!}{(p-\alpha)^{n+1}}$ \\ \hline
$t\sin\omega t$ & $\dfrac{2p\omega}{(p^2+\omega^2)^2}$ \\ \hline
$t\cos\omega t$ & $\dfrac{p^2-\omega^2}{(p^2+\omega^2)^2}$ \\ \hline
$t\sh\omega t$ & $\dfrac{2p\omega}{(p^2-\omega^2)^2}$ \\ \hline
$t\ch\omega t$ & $\dfrac{p^2-\omega^2}{(p^2-\omega^2)^2}$ \\ \hline
$e^{\alpha t}\sh\omega t$ & $\dfrac{\omega}{(p-\alpha)^2-\omega^2}$ \\ \hline
$e^{\alpha t}\ch\omega t$ & $\dfrac{p-\alpha}{(p-\alpha)^2-\omega^2}$ \\ \hline
\end{tabular}\\\vspace*{0.5cm}
Теоремы операционного исчисления\\\vspace*{0.5cm}
\begin{tabular}{|c|c|}
\hline
$f(t-\tau)$ & $F(p)e^{-p\tau}$ \\ \hline
$tf(t)$ & $-F'(p)$ \\ \hline
$\dfrac{f(t)}{t}$ & $\int\limits_p^\infty F(p)dp$ \\ \hline
$f_1(t)*f_2(t)$ & $F_1(p)F_2(p)$ \\ \hline
$f(t)e^{\alpha t}$ & $F(p-\alpha)$ \\ \hline
$f'(t)$ & $pF(p)-f(0)$ \\ \hline
$\int\limits_0^t f(\tau)d\tau$ & $\dfrac{F(p)}{p}$ \\ \hline
\end{tabular}
\end{center}
\end{document}
